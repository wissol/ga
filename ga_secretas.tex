\chapter{Palabras Secretas de Grandes Aventuras}

\varhrulefill

\section{Niveles de calidad}

En \emph{Grandes Aventuras} todas las cosas se clasifican en cinco grados de calidad de A, B, C, D y E. Algo ordinario, de calidad normal, más bien barato es C; como lo que compras rutinariamente en el supermercado. A es lo mejor de lo mejor y E significa un trasto que falla más que funciona.

Un príncipe sólo usa ropas de nivel A, a menos que las cambie por los harapos de grado E de un mendigo, como en el cuento de Mark Twain. Un barco de nivel E siempre tiene vías de agua y los marineros rezan porque no les alcance una tormenta. 

\section{Dados}

En \emph{Grandes Aventuras} jugamos con dados de seis caras. Sin embargo, podemos clasificarlos en cuatro clases, simplemente por las diferentes maneras de leerlos.

\begin{description}

\item[Dados Malos (dm):]\index{dados}
Cuando tiras dados malos ignoras cualquier resultado mayor que 4. Por tanto, si te sale un 5 ó un 6 en el dado lo lees como si te hubiera salido un 4.

\item[Dados débiles (dd):]
Cuando tiras dados débiles ignoras cualquier resultado mayor que 5. Por tanto, si te sale un un 6 en el dado lo lees como si te hubiera salido un 5.

\item[Dados normales (dn):]
Los dados normales se leen normalmente; un 1 es un 1, un 2 es un 2, un 6 es un 6, etc. 

\item[Dados fuertes (df):] 
Cuando tiras dados fuertes ignoras cualquier resultado menor que 3 Por tanto, si te sale un 1 ó un 2 en el dado lo lees como si te hubiera salido un 3.

\item[Nota Importante: ]
Si no se dice nada especial supón que las reglas se refieren a dados normales; aunque procuraré dejarlo claro siempre que pueda.

\end{description}

\minisec{Consejos con los dados}

\begin{itemize}
\item Usa dados de colores diferentes para cada tipo de dado. Por \emph{ejemplo}: negro para dados fuertes, blanco para normales, rojo para débiles y azul para los malos. Aunque el color de los dados no es importante, sí lo que es que una vez te decidas por un color, los mantengas así.
\item Aunque puedes jugar con un solo dado, si te es posible usa doce dados, tres por cada color.
\item Si no tienes suficientes dados de cuatro colores, lanzálos por orden, primero los malos, después los débiles y así sucesivamente.

\end{itemize}

\section{Otras palabras importantes}

\begin{description}

\item{Atributo:} 
Las categorías que definen las capacidades de un personaje: Fuerza, Agilidad, Destreza, Salud, Educación, Ojos y Oídos y Carisma.

\item{Máster:} 
La persona encargada de interpretar las reglas, narrar las aventuras, manejar los personajes no jugadores y asegurarse que todo el mundo se divierta.

\item{Personaje no jugador o PNJ:} 
Todo personaje, animal, monstruo, espíritu o fantasma que no es jugado por ningún jugador. La mayoría de los personajes no jugadores son neutrales pero pueden ser amigos o enemigos. En cualquier caso, el Director de Juego los controla a todos.

\item{Personaje jugador o PJ:} 
Cualquier personaje llevado por un jugador; normalmente sólo uno por personaje.

\item{Pillerías:}
Puntos que puedes usar para pedir un favor al máster, como dejarte repetir esa tirada o sobrevivir a esa caída. \enquote{\dots es que, en el último momento, me agarré a una rama y no me hice tanto daño y\dots}

\end{description}