\capitulo{La máquina del juego}

\begin{quotation}

\noindent El máster describe la escena: --\enquote{Entráis en el callejón con pasos
 leves y ojos concentrados, temiendo lo que pudiera estar acechando entre 
 las sombras. El aire rebosa con la putridez de los montones de fruta abandonada, 
 hace días al sol. ¿Por qué nadie la habra recogido todavía?}

Raúl, uno de los jugadores, --su personaje es Frank un limpiabotas de trece años, 
decide tomar la iniciativa. --\enquote{Disparo a la fruta con mi tirachinas 
y si se mueve algo, ya veremos lo que hacemos}

El máster no tiene dudas que Frank tendrá éxito y no le hace tirar los dados. --\enquote{
Vale, Raúl, tu personaje, le da a uno de los montones de fruta. Inmediatamente
salta un gato, echando espumajos por la boca, enfermo de rabia. ¿Qué hacéis?
}

La respuesta del grupo es unánime --\enquote{¡Corremos!} 

Ahora el máster no está seguro de si los personajes conseguirán escapar del gato.
Es una situación más bien caótica en la que cualquier cosa podría pasar. Para esos
casos tenemos el calibrador del destino, que te \emph{ayudarán} a decidir quién puede 
hacer qué acción y cuáles serán sus consecuencias.

\end{quotation}

\section{Regla General}

Una prueba\index{prueba} del \emph{Calibrador del destino} 
permite determinar si un personaje ha tenido éxito o ha fracasado 
en su intento de realizar una acción. La siguiente lista resume 
cómo resolvemos una prueba de una manera formal, seguramente mucho 
más de la que luego váis a aplicar en la práctica.

\begin{enumerate}
\item El jugador explica al máster la acción que su personaje está intentando.
\item El máster determina las posibles consecuencias de la acción y, 
si es necesario, se las explica al jugador. Por ejemplo, si un \emph{pj} intenta disparar a una rata con un tirachinas las consecuencias podrían ser un fallo o un acierto, con el consiguiente herida para la rata.
\item El jugador confirma que quiere intentar la acción
\item El máster decide si es necesario lanzar los dados o, si, por obvio, cuál es el resultado de la acción.
\item El máster determina luego qué atributo es el más apropiado para la acción.
\item Al mismo tiempo decide cuál es el \emph{Número Objetivo}\index{número objetivo}.
\item Sabiendo todo esto, el jugador lanza los dados. Si el total es menor que el número objetivo habrá fracasado y tendrá que afrontar las consecuencias de la acción. En caso contrario habrá tenido éxito.

\end{enumerate}

\section{El número objetivo}
El máster es el único que puede determinar cuál es el número objetivo --Desde ahora, representado por el signo \#, seguido del número-- \index{número objetivo} para cada acción. Se guía para ello de la dificultad aparente de la acción. Por ejemplo un caso normal sería \#10, --es decir un número objetivo de 10--.

\begin{table}[h]
\centering
\begin{tabular}{ccccc}
\toprule
Dificultad&Trivial&Fácil&Normal&Difícil\\\midrule
Número Objetivo&\#7 ó menos&\#8 ó \#9&\#10 a \#11&\#12 ó más\\\midrule
\bottomrule
\end{tabular}
\caption{Número objetivo}\index{número objetivo}
\end{table}

\subsection*{Cuando \emph{no} usamos los dados}

\begin{description}
\item[Acciones Imposibles] La acción es prácticamente imposible para cualquiera, como intentar volar agitando los brazos. En ese caso la acción fracasa, sin importar lo que saques en los dados ni los puntos de pillerías\index{pillerías} que gastes.
\item[Acciones Triviales] En general las acciones tan fáciles tienen siempre éxito y solo deberían tirarse si el personaje está bajo mucha presión, --como en medio de una batalla. No hace falta tirar por Agilidad cada vez que un personaje intenta montar en bicicleta o correr por el parque.
\end{description}

\section{Pifias y Éxitos espectaculares}

Cuando jugamos, en la mayoría de los casos nos basta saber si nuestra acción ha tenido éxito o no. Sin embargo, el máster puede usar la diferencia entre lo que se sacó en los dados y el número objetivo como una forma de evaluar el grado de éxito, es decir, lo bien (o mal) con el que el personaje se desenvolvió, si la pifió o, si por el contrario, tuvo un éxito espectacular. Siempre a criterio del máster, un resultado cuatro puntos inferior al número objetivo podría ser una pifia, mientras que si fuera cuatro puntos superior, podría ser un éxito espectacular.

La interpretación del grado de éxito queda también a criterio del máster. En muchos casos puede ignorarse totalmente, en otros puede cambiar completamente las consecuencias de la acción. Veamos un ejemplo de cada caso.

\begin{quotation}
Un viejo científico se está escondiendo de un esbirro que busca su pócima secreta. El máster determina que el número objetivo del esbirro para encontrar al científico es de \#10, pero los dados le son desafortundados sacando un total de \#5 en sus tres dados. En este caso el máster simplemente dice que el esbirro no encuentra al científico y se marcha.
\end{quotation}
\begin{quotation}
Un jefe nativo intenta resistir el avance de una fuerza colonial. Sus tropas están siendo diezmadas por los fusiles modernos, pero en medio del combate avista al coronel europeo. Con un supremo esfuerzo, a una distancia casi imposible para una lanza, arroja su jabalina. El máster determina que el número objetivo es \#16, y el valeroso guerrero, obtiene un 24 entre sus cuatro dados. Dadas las circunstancias el máster decide que su lanza ha conseguido matar de un golpe al coronel, mientras el resto de los soldados huyen.
\end{quotation}

Esta regla descansa sobre el buen juicio del máster, quien siempre debe aplicarla con tiento. En particular debería tener en cuenta el tipo de escena --no deberían haber muertes en una carrera--, el tono de la partida, --las pifias deberían tener consecuencias más divertidas en una simpática búsqueda del tesoro, que una aventura para salvar a la humanidad-- y lo alocadamente que los jugadores hyaan estado jugando --si un jugador intenta saltar desde un tercer piso a un coche en marcha para demostrar su agilidad, debería sufrir más por una pifia --y tener mejores consecuencias de un éxito espectacular-- que otro que se ha visto forzado a saltar para escapar de un incendio.


\section{Competiciones}

Ahora ya sabes cómo determinar si tu personaje tiene éxito. El máster te da
un número objetivo, tiras los dados que correspondan al atributo que estás probando,
y si sacas tanto o más que el número objetivo, tienes éxito. ¿Pero qué pasa
si dos o más personajes están compitiendo?

Aunque las llamo competiciones, no me refiero solo a concursos formales. 
Podríamos estar viendo quién vende más periódicos o huyendo de un gato rabioso,
\emph{compitiendo} contra el gato, a ver si nos pilla o no. En estos casos, además del número objetivo, 
tenemos en cuenta el grado de éxito.

En otras palabras, quién saca más --y además tiene supera el número objetivo--, gana.

\begin{quotation}
Supongamos que Tim y Pip son dos jóvenes que opositan por la única beca de la
escuela de arte de New Paris. El máster determina que es una prueba de Destreza,
en el que el número objetivo deberá ser, al menos, \#14. Lamentablemente, Tim y
Pip son vendedores de periódicos, Tim obtiene un 12 con sus dados, mientras que 
Pip se queda a las puertas con un 13. Pip ha sido mejor que Tim, pero no ha cumplido
los requisitos mínimos de admisión y también fracasa.
\end{quotation}

\subsection{Modificadores}

En una competición puede que uno de los bandos tenga ventaja, justa o injusta. Por
ejemplo un corredor puede haberse preparado el día anterior y otro se haya ido de
fiesta. Un \emph{newsboy} puede haber pensado un plan de ventas y se haya ganado la
confianza de los locales, mientras que otro puede ser tan nuevo que intente vender
\enquote{Discursos de la Sociedad Tipográfica} en el barrio de pescadores. Para estos casos
el máster puede aplicar un modificador positivo o negativo a las tiradas de los
competidores, que debería estar entre $\pm1$ a $\pm3$.

