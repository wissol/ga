\chapter{Las máquinas del juego}

\section{Pruebas}

Una prueba permite determinar si un personaje ha tenido éxito o ha fracaso en su intento de realizar una acción.

\section{Pruebas avanzadas}

\section{Prueba del Destino}
% \section{This is how you play this game}

% \noindent \begin{wrapfigure}{o}{0.5\columnwidth}%
% \includegraphics[clip,width=0.4\textwidth]{NBFinalIllust/singer}

% \caption{Street Singer}


% \end{wrapfigure}%
% \lettrine{S}{}ome friends are sitting at a table; one of them is
% the game master, the rest are the players, each one playing the role
% of a player character. The game master describes the scene:

% \emph{``You enter the alley with soft feet and wild hearts, trembling
% at what could be waiting for you among the shadows. The air is filled
% with the stench of the putrid fruit piled against a corner. What could
% be hiding behind it? Ghosts are supposed to exist only in fairy tales,
% but right now you aren't so sure. Was that a step you heard? Tell
% me, what are you going to do?\textquotedbl{}}

% \emph{Rick, one of the player characters, decides to take the lead.
% ``I am going to shoot at the fruit pile with my slingshot and if
% anything moves, I say we run away like crazy.\textquotedbl{}}

% \emph{If the game master is sure about the success of an action he
% grants a success and that's that; no dice required. If on the other
% hand, the action is believed impossible, the game master forbids it. }

% \emph{In this case the game master is sure that Rick can't miss hitting
% the fruit pile. He answers ``OK, Rick\textquotedbl{}. You point and
% shoot and the stone hits the rotting fruit pile. After a split second,
% you all see a ferocious, rabid cat emerging from the fruit pile, foam
% coming out of its mouth. Now tell me, what are you going to do?\textquotedbl{}}

% \emph{The group's answer is unanimous: ``Run for it!\textquotedbl{}
% Perhaps the game master is exaggerating a bit on his description of
% the cat, but they aren't taking any chances with a rabid feline.}

% Now the game master has a small problem; will the players be able
% to outrun the cat through the maze of trash-ridden alleys? That's
% why we have rules. They help you to decide who can do what and what
% the results of those actions could be, striving to be fair, fast and
% fun.


% \section{Basic checks}


% \subsection{Choosing the skill and the dice}

% Basic checks%
% \footnote{Also known as skill tests.%
% } allow us to know whether a character has been successful when attempting
% an action. It is one of the most used rules in every game session,
% so pay attention.
% \begin{enumerate}
% \item Make sure you know what \emph{action} the player character is attempting
% and what the \emph{consequences} of that action could be. For example:
% if a player shoots a rat with his slingshot, the action would be shooting
% and the possible consequences are either a hit or a miss -- and the
% effects of that hit. The game master might want to ask the player
% to make sure he understands what he's trying to do.
% \item The game master is the only one who can determine the consequences
% of any attempted action. 
% \item The game master then determines which \emph{skill} is more appropriate
% for the \emph{action} being attempted%
% \footnote{Shooting is appropriate when the action is shooting a rat with your
% slingshot. Academics is appropriate for reading.%
% }. 
% \item \emph{If there's no appropriate skill}, the game master should choose
% an \emph{attribute} instead%
% \footnote{Suppose a newsboy finds one of the first photographic inventions and
% wants to take a picture of his friends. There is no skill for photography
% so the game master might decide that you use your Education attribute
% for this action.%
% }. Again, the game master's decision is final.
% \item Use the dice of the attribute linked to the skill, \emph{upgrading}
% or\emph{ downgrading} them:

% \begin{itemize}
% \item If the character is \emph{good at} the appropriate skill, upgrade
% the dice of the linked attribute. Weak dice become normal dice (OK
% dice), while normal dice become swell dice. 
% \item If the character is \emph{bad at} the appropriate skill, downgrade
% the dice of the linked attribute. Weak dice become fool dice and normal
% dice become weak dice. 
% \item If the character is \emph{OK at} the appropriate skill, use the dice
% of the linked attribute without any modifications.
% \end{itemize}
% \end{enumerate}
% \begin{description}
% \item [{Example:}] \emph{Allan is }good at\emph{ Sweet Talk, a skill linked
% to the Charisma attribute. He has four weak dice in his Charisma attribute.
% When he tries any Sweet Talk action, such as asking a bootblack girl
% for a date, he upgrades those four weak dice to four normal dice.}
% \end{description}

% \subsection{The target number}

% The game master then determines a target number (TN). If the player
% rolls equal or higher than the target number, he succeeds. If he rolls
% less than the target number, he fails. A task of average difficulty
% will demand a TN of 10, a very easy task could use a TN of 4, while
% a difficult task can only be achieved with a TN of 14 or more. (Refer
% to 3.2.2.1 Target Number Table.)
% \begin{description}
% \item [{Example:}] Suppose Rick is shooting a can with a slingshot. Since
% Shooting is the appropriate skill for this, the game master chooses
% it to carry out the action. Shooting is linked to Eyes \& Ears. Rick's
% character sheet shows that he has three weak dice in Eyes \& Ears;
% not so good. Yet, Rick is also \emph{good at} Shooting, so he upgrades
% those three weak dice to normal dice. The game master announces that
% hitting the can will require a TN of 11 because it's quite far away.
% Rick rolls his three normal dice and gets a 13, hitting the can right
% in the middle with a satisfying noise, watching as it falls and rolls
% along the pavement. Had Rick been \emph{bad at} Shooting, he would
% have had a problem because he would have downgraded his weak dice
% to fool dice. It's hard to roll an 11 with three fool dice unless
% you use some chits%
% \footnote{Chits are discussed in Chapter \vref{cha:Chits} %
% }. If Rick had been neither \emph{good at} or \emph{bad at} Shooting,
% he would have used the three weak dice of his Agility score and had
% some chance of hitting the can.
% \end{description}

% \subsubsection*{When to not roll dice}

% If the game master is sure you will succeed (or fail), you don't have
% to roll any dice. To speed up the game, the game master may decide
% that a player character (not a non-player character) automatically
% succeeds at an easy task. Reserve rolling dice for the important stuff
% only. This is how it goes:
% \begin{description}
% \item [{Impossible~Tasks:}] The attempt is genuinely impossible, such
% as flapping your arms to fly. The attempt fails automatically, whether
% the dice is rolled or the chits are spent.
% \item [{Trivial~Tasks:}] These are so easy, you are guaranteed to succeed.
% The game master does not need to require a die roll, except if the
% task is attempted under stress such as during a fight or in an emergency.
% \item [{Very~Easy~Tasks:}] The game master may decide that the player
% character succeeds automatically, except if the task is attempted
% under stress such as during a fight or in an emergency.
% \end{description}

% \subsubsection{Target Number Table}

% This table is a guideline for the game master when he has to determine
% a target number. You can use a TN of 5 or 7 if you think that would
% be the most appropriate for any given situation%
% \footnote{Note that average (normal) difficulty means average for an adult without
% special training. Dressing a wound would be of average difficulty
% using the Healing skill. In any case, these guidelines are not the
% Bible. The game master can change this table or disregard it altogether.%
% }. 

% \begin{longtable}{cc}
% \hline
% \toprule 
% Difficulty & Suggested Target Number\tabularnewline
% \midrule
% \hline
% \endhead
% \midrule 
% Trivial & 4\tabularnewline
% \midrule 
% Very easy & 6\tabularnewline
% \midrule 
% Easy & 8\tabularnewline
% \midrule 
% Average & 10\tabularnewline
% \midrule 
% Hard & 12\tabularnewline
% \midrule 
% Very hard & 14\tabularnewline
% \midrule 
% Experts only & 15 or more\tabularnewline
% \bottomrule
% \end{longtable}




% \subsection{The fumble }

% You fumble when you roll seven points lower than your target number,
% or you roll a 1 with each and every one of your dice%
% \footnote{It's a good thing that swell dice do not have any 1s. Plus, if you
% have three weak dice in Education and you are \emph{good at} Academics,
% it would be unlikely for you to fumble unless you face a difficult
% task.%
% }.

% A fumble could not mean anything more serious than making a fool of
% yourself before your friends; falling off your bike, tripping, or
% answering to your teacher that King George the Fifth was the first
% U.S. President. However, it could have very dear consequences, especially
% when running away from a ghost or in a fight, so avoid attempting
% actions where you could easily fumble.


% \subsection{The fail}

% You fail when you roll lower than your target number, but yet not
% so bad as in a fumble.

% A fail means you have not achieved your desired goals. If you fail
% attacking somebody in a fight, you just miss your enemy, if you fumble,
% your own strength could make you fall down, face first. Still, failing
% can be tragic in some situations, like if you are trying to jump from
% one window to the next.


% \subsection{The success}

% You achieve success when you roll equal or higher than your target
% number, but yet you do not get a spectacular success.

% A success means you have reached your desired goals, which in most
% cases would be good enough.


% \subsection{The spectacular success}

% The spectacular success is the opposite of the fumble. \emph{You get
% a spectacular success }when you roll seven points\emph{ }over the
% target number\emph{, or you roll a 6 with each and every one of your
% dice}.

% A spectacular success gives you additional benefits; you'll sell newspapers
% much faster (giving you time to do something else or just sell more
% papers), you don't only know about George Washington, but you can
% tell your teacher the size of his shoes. In an emergency, a spectacular
% success could save the day.


% \subsubsection*{In short}
% \begin{itemize}
% \item The player tells the game master what action his character is attempting.
% \item The game master determines the appropriate skill to carry out that
% action. If there is no appropriate skill, he chooses an attribute.
% \item If the character is \emph{good at} the skill, he upgrades%
% \footnote{Normal dice become swell dice and weak dice become normal dice.%
% } the dice of his linked attribute. 
% \item If the character is \emph{bad at} the skill, he downgrades%
% \footnote{Normal dice become weak dice and weak dice become fool dice.%
% } the dice of his linked attribute.
% \item If the character is not \emph{OK at }the skill or is using an attribute,
% he rolls the dice of the linked attribute without upgrading or downgrading
% them.
% \item The game master determines the target number of the action as he sees
% fit%
% \footnote{A TN of 10 is average, 14 is very hard and 6 is very easy.%
% }. 
% \item The player rolls his dice, adding the results. He succeeds if the
% total is equal to or greater than the target number, and fails if
% the total is less than the target number.
% \item If the total is 7 points less than the TN or all your dice show 1s,
% it's a fumble.
% \item If the total is 7 points greater than the TN or all your dice show
% 6s, it's a spectacular success.
% \item Fumbles carry additional penalties, while a spectacular success provides
% extra benefits.
% \end{itemize}

% \section{Competitive tasks}

% Now you know how to test if a character succeeds at any given task.
% You ask the game master for a target number, you roll your dice, add
% them up, and if you get as much as the target number or higher, you
% succeed. What if two or more characters are competing to achieve the
% same task? 

% We could be trying to see who can run faster or sell the last newspaper
% to the last client of the day. Each character rolls his dice and the
% highest result wins. If two or more characters roll the same result
% they perform equally well.

% If the task was hard enough without competition, the game master might
% assign a minimum target number. If no character achieves this minimum
% they all fail. 
% \begin{quote}
% Let's suppose Tim and Pip are both taking an examination for the chance
% to earn a scholarship at the New Paris Atelier School of Arts. There's
% only one remaining scholarship and the school demands a minimum level
% of skill for admittance. The game master determines both competitors
% will need a TN of 15 or greater. Should they both fail, none of them
% will receive the scholarship. 
% \end{quote}
% Those who fumble (TN - 7), lose automatically. Those who roll a spectacular
% success (TN + 7), double their total. 


% \subsubsection*{In short}
% \begin{itemize}
% \item For a competitive task, every character rolls his dice and the highest
% total wins. 
% \item The game master \emph{may} assign a minimum target number.
% \item A fumble will lose automatically, while a spectacular success will
% double the totals.
% \item If both characters fail or fumble, there will be no winner.
% \end{itemize}

% \section{With a little help from my friends}

% What if two or more characters are trying to help each other to achieve
% the same task? First of all, the game master must make sure that teamwork
% does indeed help in the given situation. Athletics or shooting are
% examples of skills that are often individual in nature, while lifting
% a carriage from the body of an unfortunate newsboy does indeed benefit
% from as many arms as possible.

% The game master determines the target number and one of the characters
% is chosen to be the leader. The other characters are considered assistants.
% The leader rolls his dice normally. The assistants roll theirs too,
% but their results are halved, rounded up, and they then add all their
% points to the grand total.

% The game master may also assign a maximum number of helpers. 
% \begin{quote}
% Let's say that Tim and Pip are carrying a huge basket filled with
% eggs, hopefully without breaking any. The game master determines this
% is a feat of Strength and assigns a TN of 10. That's bad news for
% Tim (STR two weak dice) and Pip (STR two weak dice), so they decide
% to cooperate. Tim is chosen to be the leader and he rolls a 5 and
% a 3, for a total of 8. Pip, the helper, rolls a 4 + 3 = 7, halved
% to 3.5, rounded up to 4. The grand total is 8 + 4 = 12, which is greater
% than the TN of 10, so they succeed.
% \end{quote}

% \subsubsection*{Fumble and spectacular success}

% Any player who rolls a 1 on all of his dice causes the whole group
% to fumble. The group can also fumble if the grand total is 7 under
% the target number.

% The group achieves a spectacular success if the grand total is 7 over
% the target number, or if the leader rolls a 6 with all their dice
% and nobody fumbles; if anybody fumbles, the whole group fumbles.


% \subsubsection*{In short}
% \begin{itemize}
% \item The game master must agree that the task can be done in a group.
% \item One of the characters is chosen to be the leader, the rest are considered
% assistants.
% \item The leader rolls his dice and adds them up normally.
% \item The assistants roll their dice, but totals are halved and rounded
% up.
% \item Add the leader's and the assistants' rolls to find the grand total.
% \item If the grand total is equal to or greater than the target number,
% the group succeeds.
% \item If anybody fumbles, the whole group fumbles.
% \item If the grand total is lower than the TN - 7, it's a fumble.
% \item If the grand total is greater than the TN + 7, it's a spectacular
% success.
% \end{itemize}

% \section{Quick and dirty or slow and neat}

% The game assumes that most tasks are instantaneous or can be accomplished
% in less than two minutes. However, there are actions that require
% time, which is in short supply in an adventure. What then? The game
% master assigns a time frame to the action, according to his own good
% sense. If the player wants to speed up their action \emph{just a little
% bit} the game master should raise the TN by 1. If the player wants
% to do the action in half the time, the game master should double the
% target number.

% If the players are happy with slowing down their actions \emph{just
% a little bit} the game master may lower the TN by 1. If the players
% want to act slowly and safely the game master should lower the TN
% by 3.
% \begin{quote}
% \begin{onehalfspace}
% Suppose the players need to know how much time it would take them
% to do their homework%
% \footnote{Normally, a player doesn't roll for doing homework, but these players
% are trying to get to college eventually, so they want great grades.%
% } for the Newsboy Lodge Night School. The game master decides that
% it would normally take two hours. The players are not happy about
% that, so they ask to speed up that time by one third for a final time
% of one hour and 20 minutes. The game master thus determines that normally
% to get an A in that homework they would require a TN of 8, but since
% they are speeding up, he raises the TN by 1 to a total of 9.\end{onehalfspace}

% \end{quote}

% \subsubsection*{In short}
% \begin{itemize}
% \item If the players want to speed up their actions \emph{just a little
% bit} the game master should raise the TN by 1. 
% \item If the players want to act in half the time, the game master should
% double the target number.
% \item If the players are happy with slowing down their actions \emph{just
% a little bit} the game master may lower the TN by 1. 
% \item If the players want to act slow and safe the game master should lower
% the TN by 5.
% \end{itemize}

% \section{Try again}

% Most actions can be attempted more than once. In each case the game
% master must be satisfied that the characters still have the time,
% the materials and any other requirement. You can't try again with
% a fine Chinese vase that fell down, rolled and broke into smithereens.
% You can try again with a drawing that went wrong, provided you have
% a pencil and an eraser. 

% The game master will choose to use the same target number from the
% first attempt \emph{unless there is a good reason} to change it. A
% second or third attempt at escaping from the watch of kidnappers will
% be harder as the criminals will have been alerted. On the other hand,
% a second attempt at solving a mathematics question should be easier.

% However, if you try again but fail, you must pay 1 chit for the opportunity
% to attempt that same task again. 
% \begin{quote}
% Let's say Bobby is trying to pick the lock of a door to a room where
% he is locked in. Unfortunately, he's \emph{bad at} Locks and has only
% two weak dice in Agility. Since he is \emph{bad at} Locks his two
% weak dice are downgraded to two fool dice. The game master asks him
% for a TN of 8; difficult with just two fool dice. The player decides
% to try again and the game master concedes he has a better chance to
% pick the lock now that he's familiar with it, and lowers the TN to
% 7. Poor Bobby rolls his dice again and fails. Now, Bobby has heard
% that the corpse with whom he shares the room where he's locked in,
% is from Transylvania and he does not want to risk any legends of vampires.
% So, he pays 1 chit for another go with a TN lowered to 6. This time
% he succeeds%
% \footnote{As you will learn in Chapter \vref{cha:Chits}, Bobby could have used
% 1 chit for a re-roll in any of his earlier attempts. He chose not
% to do so because it was 9 a.m. (when a vampire sleeps), and he needed
% to lower the target number.%
% }. 
% \end{quote}
% Make sure the players understand that to try again is not the same
% as re-rolling. When you re-roll, you are rolling your dice again for
% the same action. If you try again you are making a new attempt%
% \footnote{If the characters doing their homework in the earlier example decided
% to try again, they would have had to erase everything they'd already
% done and start over.%
% }.


% \subsubsection*{In short}
% \begin{itemize}
% \item When a character wants to try again, he attempts a task that he has
% already tried and failed at.
% \item The game master must be assured that the action is still possible. 
% \item Have the characters used up the materials, tools and fuel, etc.?
% \item The game master can adjust the target number of the action if any
% successive attempt makes the action harder or easier.
% \item If the player fails three times in a row, he must pay 1 chit for the
% opportunity to have a fourth attempt.
% \end{itemize}

% \section{Fallen dice}

% Any die that falls off the table \emph{when testing for a skill or
% attribute} is read as a 1. Both players and the game master should
% be careful when throwing their dice, for their own benefit.


% \section{Detailed skill list}

% As promised, here is the detailed skill list, organized by each skill's
% linked attribute%
% \footnote{There are no skills linked to Health.%
% }. This list provides reference and guidelines for the game master
% and players, so you don't have to memorize it%
% \footnote{Please note that the Language skill works differently. In fact, it's
% so special it has its own section.%
% }.


% \subsection{Strength skills}
% \begin{description}
% \item [{Fisticuffs\,(STR):}] Fisticuffs is the appropriate skill to use
% when you get into a fight, as long as you fight with your own body
% or use hand weapons such as a stick or a sword. You can use this skill
% in a friendly wrestling competition or while mock boxing. Don't use
% this skill for throwing a stone or using a slingshot. That action
% is better reserved for Throwing and Shooting.\\
% You can usually try again as long as you are not in a fight. However,
% make each attempt harder by raising the TN by 1 if you don't rest
% for at least five minutes.%
% \footnote{You'll find more details on how to use Strength in Chapter \vref{cha:Fights,-Chases-and}%
% }
% \item [{Throwing\,(STR):}] Throwing is the skill of choice when you are
% throwing anything, either a carefully crafted javelin or throwing
% a knife to the most humble stone. \\
% You can try again as long as you don't run out of objects to throw.
% \end{description}

% \subsection{Agility skills}
% \begin{description}
% \item [{Athletics\,(AGI):}] Athletics is the appropriate skill for any
% action dealing with a sport or similar activity, not already covered
% by more specific skills such as Climbing or Riding. So, when a character
% is trying to climb a wall, play baseball or football, or jump over
% a fence, this skill would, under most circumstances, be helpful.\\
% A fumble in this skill can mean that you trip over yourself, perhaps
% suffering a few cuts and bruises%
% \footnote{See the Damage Table \vpageref{sub:TDamage-Scale.}%
% }.\\
% Athletics is also appropriate when the characters are trying to run
% away from a criminal on foot, when participating in a race with their
% friends, or when trying to arrive on time to catch the Elevated Train
% or Subway. \\
% You can usually try again unless you are involved in a competition.
% \item [{Climbing\,(AGI):}] Climbing might be your last hope of escape
% when cornered in a narrow alley. It's also useful when trying to sneak
% into (or out of) a building, and to survive the most treacherous passages
% of the New Paris sewers.\\
% You can usually try again as long as you are not in a fight, but make
% each attempt harder by raising the TN by 1 if you don't rest for at
% least five minutes.
% \item [{Dodging\,(AGI):}] Dodging is an ability mostly used in a fight
% to avoid being hit, as discussed in Chapter \vref{cha:Fights,-Chases-and},
% but it can also come in handy to avoid being run over by a train,
% carriage or a horse. \\
% You cannot try again to dodge the same attack.
% \item [{Locks\,(AGI):}] This skill comprises the ability to open locks
% and safes, \emph{provided you have the proper tools}. Opening the
% door of a poor tenement house is relatively easy, requiring a TN of
% 12. A basic safety lock requires at least a TN of 13; a safe would
% require at least a TN of 15. With improvised tools, increase the TN
% by 3 to a total of 5.\\
% A fumble means you have damaged your tools, making them unusable even
% after they are properly repaired. You can usually try again unless
% you don't fumble.
% \item [{Pickpockets\,(AGI):}] This skill is useful for the obvious task
% of stealing wallets, watches and expensive handkerchiefs while the
% poor victim doesn't notice. Stealing from an unwary old lady should
% demand a TN of 12, but who would be such a crook? However, stealing
% from a wary policeman commands a TN of 16 or more. For other victims,
% the game master should use his good sense.\\
% A failure in this task indicates that the victim notices before you
% get to steal anything. 
% \item [{Riding\,(AGI):}] Riding includes the capacity to drive a bicycle,
% for example, and the ability to maintain it. The Mechanics skill (not
% Crafts) is also adequate to care for a bicycle, but not to ride it.
% As with Climbing and Athletics, a fumble in this skill will usually
% mean that you fall off your bike, perhaps suffering a few cuts and
% bruises. \\
% You can usually try again unless you are involved in a competition
% or you are being chased.
% \item [{Stealth\,(AGI):}] This skill comes in handy when playing hide-and-seek,
% especially if the one who seeks you is little Pete and his pet Rottweiler.
% In short, it's good for both hiding and moving silently without being
% detected. \\
% You cannot usually try again once you've been discovered.
% \item [{Swimming\,(AGI):}] Only characters who are \emph{good at }Swimming
% may swim in relatively calm waters without having to roll for this
% skill. Everybody else must roll for Swimming when they are swimming
% in water deeper than their height, with a TN of 8 for rough waters.
% \\
% Failure means that the player character cannot advance and must struggle
% to keep himself afloat. \\
% A fumble puts the character in immediate danger of drowning. In this
% case, the player character must struggle for his life, rolling once
% every minute. Any further failure means the player character would
% drown in five minutes, unless rescued. 
% \item [{Vehicles\,(AGI):}] The Vehicles skill is useful to operate and
% maintain any land vehicle except a bicycle. This includes trains and
% stagecoaches, but not riding a horse.\\
% You can try again unless you are being chased or are undertaking some
% competition.
% \end{description}

% \subsection{Education skills}
% \begin{description}
% \item [{Academics\,(EDU):}] Academics is a glorified term for school stuff.
% History, Biology, Botanics, Philosophy and in general, all school,
% high school and university subjects not covered by more specific skills
% belong to Academics. Use this skill when a character wants to know
% the meaning of some word, the poisonous effects of some plant or do
% research in a library. \\
% You can \emph{only} try agai'' if \emph{your character} is aware
% that he has failed, he can consult reference works, or ask a learned
% person such as a teacher.
% \item [{Crafts\,(EDU)~and~Mechanics\,(EDU):}] Crafts covers all sorts
% of traditional crafts, maintenance and repairs not related to electricity
% or mechanics; sewing, carpentry, masonry, drawing, painting, etc.,
% are covered by this skill. Crafts can be a pretty useful skill in
% the hands of the right player, provided there are enough tools, materials
% and time available. Repairing a train engine or some new wonder of
% science such as a horseless carriage is covered by Mechanics.\\
% A fumble at any of these skills lowers the quality of the item by
% one level.%
% \footnote{B-grade clothes downgrade to C, C-grade to D, D-grade to E and with
% E, well, you better get a barrel or something.%
% }\\
% You can usually try again as long as the materials are not used up
% by the attempt%
% \footnote{You can try again as long as you have paper or an eraser handy.%
% }. 
% \item [{Healing\,(EDU):}] Healing covers everything from dressing the
% leg of a puppy to minor surgery. Full details on how to use this skill
% are covered on \vref{sub:Healing} 
% \item [{Streetwise\,(EDU):}] Streetwise makes you aware of the places
% and people you ought to know (or avoid) in New Paris.\\
% You can't usually try again with this skill; if you don't know, then
% you don't know.
% \end{description}

% \subsection{Eyes \& Ears skills}
% \begin{description}
% \item [{Observation\,(E\&E):}] Observation is a great skill to discover
% someone trying to hide from you, trying to read a note in the dark
% or recognizing a face among the crowd.
% \item [{Shooting\,(E\&E):}] Shooting includes the skill to use and maintain
% any weapon that shoots a missile of some sort.
% \end{description}

% \subsection{Charisma skills}
% \begin{description}
% \item [{Performance\,(CHA):}] Being \emph{good at} Performance helps you
% to perform on the streets for a few dimes, juggle or even run a puppet
% show for children. 
% \item [{Languages\,(CHA):}] Languages is in fact, two skills in one: a)
% how many languages you know well enough to have a basic conversation;
% and b) communicating through mime and gestures. Note that standard
% sign language is considered a language; not mime and gestures.\\
% A character who is \emph{good at} Languages is entitled to test this
% skill the first time he needs to use a foreign language. If the test
% is successful the character can use that language. \\
% Failure means the character is not able to communicate in that language
% and will be limited to mime and gestures. The game master should assign
% a target number using his good sense, taking into account the character's
% background. A TN of 12 is suggested for common and easy languages;
% while rare or hard tongues should get higher numbers. Note that this
% test cannot be repeated without spending chits.\\
% A character who is \emph{bad at }or\emph{ OK at} Languages can only
% speak English.\\
% You can only try again with this skill after spending at least one
% week immersed in the language or after one month of intensive training.
% \item [{Sweet~Talk\,(CHA):}] This skill is used to convince people to
% agree with your opinions and make them do what you want them to do.
% It can even be used, God forbid, to tell some untruthful story to
% a policeman who may have just found you with a gold watch that just
% \emph{accidentally} fell into your pocket. The target number varies
% wildly; the story about a watch that \emph{accidentally} fell into
% a newsboy's pocket should command a TN of 15, at the very least.\\
% A fumble means the other person is angry enough that he would not
% pay attention to your words for one full week, and it could have the
% reverse consequences that you intended.\\
% You can try again as long as you don't fumble, but don't forget that
% each attempt raises the TN by 1.
% \end{description}