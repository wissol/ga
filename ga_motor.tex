\capitulo{Las máquinas del juego}
%\chapter{Las máquinas del juego}
%\varhrulefill
\begin{quotation}

El máster describe la escena: --\enquote{Entráis en el callejón con pasos
 leves y ojos concentrados, temiendo lo que pudiera estar acechando entre 
 las sombras. El aire rebosa con la putridez de los montones de fruta abandonada, 
 hace días al sol. ¿Por qué nadie la habra recogido todavía?}

Raúl, uno de los jugadores, --su personaje es Frank un limpiabotas de trece años, 
decide tomar la iniciativa. --\enquote{Disparo a la fruta con mi tirachinas 
y si se mueve algo, ya veremos lo que hacemos}

El máster no tiene dudas que Frank tendrá éxito y no le hace tirar los dados. --\enquote{
Vale, Raúl, tu personaje, le da a uno de los montones de fruta. Inmediatamente
salta un gato, echando espumajos por la boca, enfermo de rabia. ¿Qué hacéis?
}

La respuesta del grupo es unánime --\enquote{¡Corremos!} 

Ahora el máster no está seguro de si los personajes conseguirán escapar del gato.
Es una situación más bien caótica en la que cualquier cosa podría pasar. Para esos
casos tenemos el calibrador del destino, que te \emph{ayudarán} a decidir quién puede 
hacer qué acción y cuáles serán sus consecuencias.

\end{quotation}

\section{El Calibrador del Destino}

\subsection{Regla General}

Una prueba\index{prueba} del \textsc{Calibrador del destino} permite determinar si un personaje ha tenido éxito o ha fracasado en su intento de realizar una acción. La siguiente lista resume cómo resolvemos una prueba de una manera formal, seguramente mucho más de la que luego váis a aplicar en la práctica.

\begin{enumerate}
\item El jugador explica al máster la acción que su personaje está intentando.
\item El máster determina las posibles consecuencias de la acción y, si es necesario, se las explica al jugador. Por ejemplo, si un \textsc{pj} intenta disparar a una rata con un tirachinas las consecuencias podrían ser un fallo o un acierto, con el consiguiente herida para la rata.
\item El jugador confirma que quiere intentar la acción
\item El máster decide si es necesario lanzar los dados o, si, por obvio, cuál es el resultado de la acción.
\item El máster determina luego qué atributo es el más apropiado para la acción.
\item Al mismo tiempo decide cuál es el \textsc{Número Objetivo}\index{número objetivo}.
\item Sabiendo todo esto, el jugador lanza los dados. Si el total es menor que el número objetivo habrá fracasado y tendrá que afrontar las consecuencias de la acción. En caso contrario habrá tenido éxito.
%\item ¿Subir y bajar?
\end{enumerate}

\subsection{El número objetivo}
El máster es el único que puede determinar cuál es el número objetivo --Desde ahora, representado por el signo \#, seguido del número-- \index{número objetivo} para cada acción. Se guía para ello de la dificultad aparente de la acción. Por ejemplo un caso normal sería \#10, --es decir un número objetivo de 10--.

\begin{table}[h]
\centering
\begin{tabular}{ccccc}
\toprule
Dificultad&Trivial&Fácil&Normal&Difícil\\\midrule
Número Objetivo&\#7 ó menos&\#8 ó \#9&\#10 a \#11&\#12 ó más\\\midrule
\bottomrule
\end{tabular}
\caption{Número objetivo}\index{número objetivo}
\end{table}

\subsubsection*{Cuando \emph{no} usamos los dados}

\begin{description}
\item[Acciones Imposibles] La acción es prácticamente imposible para cualquiera, como intentar volar agitando los brazos. En ese caso la acción fracasa, sin importar lo que saques en los dados ni los puntos de pillerías\index{pillerías} que gastes.
\item[Acciones Triviales] En general las acciones tan fáciles tienen siempre éxito y solo deberían tirarse si el personaje está bajo mucha presión, --como en medio de una batalla. No hace falta tirar por Agilidad cada vez que un personaje intenta montar en bicicleta o correr por el parque.
\end{description}

\subsection{Pifias y Éxitos espectaculares}

Cuando jugamos, en la mayoría de los casos nos basta saber si nuestra acción ha tenido éxito o no. Sin embargo, el máster puede usar la diferencia entre lo que se sacó en los dados y el número objetivo como una forma de evaluar el grado de éxito, es decir, lo bien (o mal) con el que el personaje se desenvolvió, si la pifió o, si por el contrario, tuvo un éxito espectacular. Siempre a criterio del máster, un resultado cuatro puntos inferior al número objetivo podría ser una pifia, mientras que si fuera cuatro puntos superior, podría ser un éxito espectacular.

La interpretación del grado de éxito queda también a criterio del máster. En muchos casos puede ignorarse totalmente, en otros puede cambiar completamente las consecuencias de la acción. Veamos un ejemplo de cada caso.

\begin{quotation}
Un viejo científico se está escondiendo de un esbirro que busca su pócima secreta. El máster determina que el número objetivo del esbirro para encontrar al científico es de \#10, pero los dados le son desafortundados sacando un total de \#5 en sus tres dados. En este caso el máster simplemente dice que el esbirro no encuentra al científico y se marcha.
\end{quotation}
\begin{quotation}
Un jefe nativo intenta resistir el avance de una fuerza colonial. Sus tropas están siendo diezmadas por los fusiles modernos, pero en medio del combate avista al coronel europeo. Con un supremo esfuerzo, a una distancia casi imposible para una lanza, arroja su jabalina. El máster determina que el número objetivo es \#16, y el valeroso guerrero, obtiene un 24 entre sus cuatro dados. Dadas las circunstancias el máster decide que su lanza ha conseguido matar de un golpe al coronel, mientras el resto de los soldados huyen.
\end{quotation}

Esta regla descansa sobre el buen juicio del máster, quien siempre debe aplicarla con tiento. En particular debería tener en cuenta el tipo de escena --no deberían haber muertes en una carrera--, el tono de la partida, --las pifias deberían tener consecuencias más divertidas en una simpática búsqueda del tesoro, que una aventura para salvar a la humanidad-- y lo alocadamente que los jugadores hyaan estado jugando --si un jugador intenta saltar desde un tercer piso a un coche en marcha para demostrar su agilidad, debería sufrir más por una pifia --y tener mejores consecuencias de un éxito espectacular-- que otro que se ha visto forzado a saltar para escapar de un incendio.


\subsection{Competiciones}

Ahora ya sabes cómo determinar si tu personaje tiene éxito. El máster te da
un número objetivo, tiras los dados que correspondan al atributo que estás probando,
y si sacas tanto o más que el número objetivo, tienes éxito. ¿Pero qué pasa
si dos o más personajes están compitiendo?

Podríamos estar viendo quién vende más periódicos o huyendo de un gato rabioso.
En estos casos, además del número objetivo, tenemos en cuenta el grado de éxito.
En otras palabras, quién saca más --y además tiene éxito--, gana.

\begin{quotation}
Supongamos que Tim y Pip son dos jóvenes que opositan por la única beca de la
escuela de arte de New Paris. El máster determina que es una prueba de Destreza,
en el que el número objetivo deberá ser, al menos, \#14. Lamentablemente, Tim y
Pip son vendedores de periódicos, Tim obtiene un 12 con sus dados, mientras que 
Pip se queda a las puertas con un 13. Pip ha sido mejor que Tim, pero no ha cumplido
los requisitos mínimos de admisión y también fracasa.
\end{quotation}

\subsection{Cooperación}

% \section{With a little help from my friends}

% What if two or more characters are trying to help each other to achieve
% the same task? First of all, the game master must make sure that teamwork
% does indeed help in the given situation. Athletics or shooting are
% examples of skills that are often individual in nature, while lifting
% a carriage from the body of an unfortunate newsboy does indeed benefit
% from as many arms as possible.

% The game master determines the target number and one of the characters
% is chosen to be the leader. The other characters are considered assistants.
% The leader rolls his dice normally. The assistants roll theirs too,
% but their results are halved, rounded up, and they then add all their
% points to the grand total.

% The game master may also assign a maximum number of helpers. 
% \begin{quote}
% Let's say that Tim and Pip are carrying a huge basket filled with
% eggs, hopefully without breaking any. The game master determines this
% is a feat of Strength and assigns a TN of 10. That's bad news for
% Tim (STR two weak dice) and Pip (STR two weak dice), so they decide
% to cooperate. Tim is chosen to be the leader and he rolls a 5 and
% a 3, for a total of 8. Pip, the helper, rolls a 4 + 3 = 7, halved
% to 3.5, rounded up to 4. The grand total is 8 + 4 = 12, which is greater
% than the TN of 10, so they succeed.
% \end{quote}

% \subsubsection*{Fumble and spectacular success}

% Any player who rolls a 1 on all of his dice causes the whole group
% to fumble. The group can also fumble if the grand total is 7 under
% the target number.

% The group achieves a spectacular success if the grand total is 7 over
% the target number, or if the leader rolls a 6 with all their dice
% and nobody fumbles; if anybody fumbles, the whole group fumbles.

\subsection{Prisas y pausas}

% \section{Quick and dirty or slow and neat}

% The game assumes that most tasks are instantaneous or can be accomplished
% in less than two minutes. However, there are actions that require
% time, which is in short supply in an adventure. What then? The game
% master assigns a time frame to the action, according to his own good
% sense. If the player wants to speed up their action \emph{just a little
% bit} the game master should raise the TN by 1. If the player wants
% to do the action in half the time, the game master should double the
% target number.

% If the players are happy with slowing down their actions \emph{just
% a little bit} the game master may lower the TN by 1. If the players
% want to act slowly and safely the game master should lower the TN
% by 3.
% \begin{quote}
% \begin{onehalfspace}
% Suppose the players need to know how much time it would take them
% to do their homework%
% \footnote{Normally, a player doesn't roll for doing homework, but these players
% are trying to get to college eventually, so they want great grades.%
% } for the Newsboy Lodge Night School. The game master decides that
% it would normally take two hours. The players are not happy about
% that, so they ask to speed up that time by one third for a final time
% of one hour and 20 minutes. The game master thus determines that normally
% to get an A in that homework they would require a TN of 8, but since
% they are speeding up, he raises the TN by 1 to a total of 9.\end{onehalfspace}

% \end{quote}

% \subsubsection*{In short}
% \begin{itemize}
% \item If the players want to speed up their actions \emph{just a little
% bit} the game master should raise the TN by 1. 
% \item If the players want to act in half the time, the game master should
% double the target number.
% \item If the players are happy with slowing down their actions \emph{just
% a little bit} the game master may lower the TN by 1. 
% \item If the players want to act slow and safe the game master should lower
% the TN by 5.
% \end{itemize}

\subsection{La segunda oportunidad}

% \section{Try again}

% Most actions can be attempted more than once. In each case the game
% master must be satisfied that the characters still have the time,
% the materials and any other requirement. You can't try again with
% a fine Chinese vase that fell down, rolled and broke into smithereens.
% You can try again with a drawing that went wrong, provided you have
% a pencil and an eraser. 

% The game master will choose to use the same target number from the
% first attempt \emph{unless there is a good reason} to change it. A
% second or third attempt at escaping from the watch of kidnappers will
% be harder as the criminals will have been alerted. On the other hand,
% a second attempt at solving a mathematics question should be easier.

% However, if you try again but fail, you must pay 1 chit for the opportunity
% to attempt that same task again. 
% \begin{quote}
% Let's say Bobby is trying to pick the lock of a door to a room where
% he is locked in. Unfortunately, he's \emph{bad at} Locks and has only
% two weak dice in Agility. Since he is \emph{bad at} Locks his two
% weak dice are downgraded to two fool dice. The game master asks him
% for a TN of 8; difficult with just two fool dice. The player decides
% to try again and the game master concedes he has a better chance to
% pick the lock now that he's familiar with it, and lowers the TN to
% 7. Poor Bobby rolls his dice again and fails. Now, Bobby has heard
% that the corpse with whom he shares the room where he's locked in,
% is from Transylvania and he does not want to risk any legends of vampires.
% So, he pays 1 chit for another go with a TN lowered to 6. This time
% he succeeds%
% \footnote{As you will learn in Chapter \vref{cha:Chits}, Bobby could have used
% 1 chit for a re-roll in any of his earlier attempts. He chose not
% to do so because it was 9 a.m. (when a vampire sleeps), and he needed
% to lower the target number.%
% }. 
% \end{quote}
% Make sure the players understand that to try again is not the same
% as re-rolling. When you re-roll, you are rolling your dice again for
% the same action. If you try again you are making a new attempt%
% \footnote{If the characters doing their homework in the earlier example decided
% to try again, they would have had to erase everything they'd already
% done and start over.%
% }.


% \subsubsection*{In short}
% \begin{itemize}
% \item When a character wants to try again, he attempts a task that he has
% already tried and failed at.
% \item The game master must be assured that the action is still possible. 
% \item Have the characters used up the materials, tools and fuel, etc.?
% \item The game master can adjust the target number of the action if any
% successive attempt makes the action harder or easier.
% \item If the player fails three times in a row, he must pay 1 chit for the
% opportunity to have a fourth attempt.
% \end{itemize}

% \section{Fallen dice}

% Any die that falls off the table \emph{when testing for a skill or
% attribute} is read as a 1. Both players and the game master should
% be careful when throwing their dice, for their own benefit.

\subsection{Acciones y atributos}

% \section{Detailed skill list}

% As promised, here is the detailed skill list, organized by each skill's
% linked attribute%
% \footnote{There are no skills linked to Health.%
% }. This list provides reference and guidelines for the game master
% and players, so you don't have to memorize it%
% \footnote{Please note that the Language skill works differently. In fact, it's
% so special it has its own section.%
% }.


% \subsection{Strength skills}
% \begin{description}
% \item [{Fisticuffs\,(STR):}] Fisticuffs is the appropriate skill to use
% when you get into a fight, as long as you fight with your own body
% or use hand weapons such as a stick or a sword. You can use this skill
% in a friendly wrestling competition or while mock boxing. Don't use
% this skill for throwing a stone or using a slingshot. That action
% is better reserved for Throwing and Shooting.\\
% You can usually try again as long as you are not in a fight. However,
% make each attempt harder by raising the TN by 1 if you don't rest
% for at least five minutes.%
% \footnote{You'll find more details on how to use Strength in Chapter \vref{cha:Fights,-Chases-and}%
% }
% \item [{Throwing\,(STR):}] Throwing is the skill of choice when you are
% throwing anything, either a carefully crafted javelin or throwing
% a knife to the most humble stone. \\
% You can try again as long as you don't run out of objects to throw.
% \end{description}

% \subsection{Agility skills}
% \begin{description}
% \item [{Athletics\,(AGI):}] Athletics is the appropriate skill for any
% action dealing with a sport or similar activity, not already covered
% by more specific skills such as Climbing or Riding. So, when a character
% is trying to climb a wall, play baseball or football, or jump over
% a fence, this skill would, under most circumstances, be helpful.\\
% A fumble in this skill can mean that you trip over yourself, perhaps
% suffering a few cuts and bruises%
% \footnote{See the Damage Table \vpageref{sub:TDamage-Scale.}%
% }.\\
% Athletics is also appropriate when the characters are trying to run
% away from a criminal on foot, when participating in a race with their
% friends, or when trying to arrive on time to catch the Elevated Train
% or Subway. \\
% You can usually try again unless you are involved in a competition.
% \item [{Climbing\,(AGI):}] Climbing might be your last hope of escape
% when cornered in a narrow alley. It's also useful when trying to sneak
% into (or out of) a building, and to survive the most treacherous passages
% of the New Paris sewers.\\
% You can usually try again as long as you are not in a fight, but make
% each attempt harder by raising the TN by 1 if you don't rest for at
% least five minutes.
% \item [{Dodging\,(AGI):}] Dodging is an ability mostly used in a fight
% to avoid being hit, as discussed in Chapter \vref{cha:Fights,-Chases-and},
% but it can also come in handy to avoid being run over by a train,
% carriage or a horse. \\
% You cannot try again to dodge the same attack.
% \item [{Locks\,(AGI):}] This skill comprises the ability to open locks
% and safes, \emph{provided you have the proper tools}. Opening the
% door of a poor tenement house is relatively easy, requiring a TN of
% 12. A basic safety lock requires at least a TN of 13; a safe would
% require at least a TN of 15. With improvised tools, increase the TN
% by 3 to a total of 5.\\
% A fumble means you have damaged your tools, making them unusable even
% after they are properly repaired. You can usually try again unless
% you don't fumble.
% \item [{Pickpockets\,(AGI):}] This skill is useful for the obvious task
% of stealing wallets, watches and expensive handkerchiefs while the
% poor victim doesn't notice. Stealing from an unwary old lady should
% demand a TN of 12, but who would be such a crook? However, stealing
% from a wary policeman commands a TN of 16 or more. For other victims,
% the game master should use his good sense.\\
% A failure in this task indicates that the victim notices before you
% get to steal anything. 
% \item [{Riding\,(AGI):}] Riding includes the capacity to drive a bicycle,
% for example, and the ability to maintain it. The Mechanics skill (not
% Crafts) is also adequate to care for a bicycle, but not to ride it.
% As with Climbing and Athletics, a fumble in this skill will usually
% mean that you fall off your bike, perhaps suffering a few cuts and
% bruises. \\
% You can usually try again unless you are involved in a competition
% or you are being chased.
% \item [{Stealth\,(AGI):}] This skill comes in handy when playing hide-and-seek,
% especially if the one who seeks you is little Pete and his pet Rottweiler.
% In short, it's good for both hiding and moving silently without being
% detected. \\
% You cannot usually try again once you've been discovered.
% \item [{Swimming\,(AGI):}] Only characters who are \emph{good at }Swimming
% may swim in relatively calm waters without having to roll for this
% skill. Everybody else must roll for Swimming when they are swimming
% in water deeper than their height, with a TN of 8 for rough waters.
% \\
% Failure means that the player character cannot advance and must struggle
% to keep himself afloat. \\
% A fumble puts the character in immediate danger of drowning. In this
% case, the player character must struggle for his life, rolling once
% every minute. Any further failure means the player character would
% drown in five minutes, unless rescued. 
% \item [{Vehicles\,(AGI):}] The Vehicles skill is useful to operate and
% maintain any land vehicle except a bicycle. This includes trains and
% stagecoaches, but not riding a horse.\\
% You can try again unless you are being chased or are undertaking some
% competition.
% \end{description}

% \subsection{Education skills}
% \begin{description}
% \item [{Academics\,(EDU):}] Academics is a glorified term for school stuff.
% History, Biology, Botanics, Philosophy and in general, all school,
% high school and university subjects not covered by more specific skills
% belong to Academics. Use this skill when a character wants to know
% the meaning of some word, the poisonous effects of some plant or do
% research in a library. \\
% You can \emph{only} try agai'' if \emph{your character} is aware
% that he has failed, he can consult reference works, or ask a learned
% person such as a teacher.
% \item [{Crafts\,(EDU)~and~Mechanics\,(EDU):}] Crafts covers all sorts
% of traditional crafts, maintenance and repairs not related to electricity
% or mechanics; sewing, carpentry, masonry, drawing, painting, etc.,
% are covered by this skill. Crafts can be a pretty useful skill in
% the hands of the right player, provided there are enough tools, materials
% and time available. Repairing a train engine or some new wonder of
% science such as a horseless carriage is covered by Mechanics.\\
% A fumble at any of these skills lowers the quality of the item by
% one level.%
% \footnote{B-grade clothes downgrade to C, C-grade to D, D-grade to E and with
% E, well, you better get a barrel or something.%
% }\\
% You can usually try again as long as the materials are not used up
% by the attempt%
% \footnote{You can try again as long as you have paper or an eraser handy.%
% }. 
% \item [{Healing\,(EDU):}] Healing covers everything from dressing the
% leg of a puppy to minor surgery. Full details on how to use this skill
% are covered on \vref{sub:Healing} 
% \item [{Streetwise\,(EDU):}] Streetwise makes you aware of the places
% and people you ought to know (or avoid) in New Paris.\\
% You can't usually try again with this skill; if you don't know, then
% you don't know.
% \end{description}

% \subsection{Eyes \& Ears skills}
% \begin{description}
% \item [{Observation\,(E\&E):}] Observation is a great skill to discover
% someone trying to hide from you, trying to read a note in the dark
% or recognizing a face among the crowd.
% \item [{Shooting\,(E\&E):}] Shooting includes the skill to use and maintain
% any weapon that shoots a missile of some sort.
% \end{description}

% \subsection{Charisma skills}
% \begin{description}
% \item [{Performance\,(CHA):}] Being \emph{good at} Performance helps you
% to perform on the streets for a few dimes, juggle or even run a puppet
% show for children. 
% \item [{Languages\,(CHA):}] Languages is in fact, two skills in one: a)
% how many languages you know well enough to have a basic conversation;
% and b) communicating through mime and gestures. Note that standard
% sign language is considered a language; not mime and gestures.\\
% A character who is \emph{good at} Languages is entitled to test this
% skill the first time he needs to use a foreign language. If the test
% is successful the character can use that language. \\
% Failure means the character is not able to communicate in that language
% and will be limited to mime and gestures. The game master should assign
% a target number using his good sense, taking into account the character's
% background. A TN of 12 is suggested for common and easy languages;
% while rare or hard tongues should get higher numbers. Note that this
% test cannot be repeated without spending chits.\\
% A character who is \emph{bad at }or\emph{ OK at} Languages can only
% speak English.\\
% You can only try again with this skill after spending at least one
% week immersed in the language or after one month of intensive training.
% \item [{Sweet~Talk\,(CHA):}] This skill is used to convince people to
% agree with your opinions and make them do what you want them to do.
% It can even be used, God forbid, to tell some untruthful story to
% a policeman who may have just found you with a gold watch that just
% \emph{accidentally} fell into your pocket. The target number varies
% wildly; the story about a watch that \emph{accidentally} fell into
% a newsboy's pocket should command a TN of 15, at the very least.\\
% A fumble means the other person is angry enough that he would not
% pay attention to your words for one full week, and it could have the
% reverse consequences that you intended.\\
% You can try again as long as you don't fumble, but don't forget that
% each attempt raises the TN by 1.
% \end{description}

\section{El Telar del Destino}



