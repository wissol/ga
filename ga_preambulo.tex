\documentclass[12pt,spanish,a4paper,DIV=7,twoside=false]{scrbook}
% fuente base 12, en Español, din A4, la medida del área de texto que prefiero, a una sola cara,

% ------ Línea de 80 caracteres -----------------------------------------------


% ------ de borrador ------------

\usepackage{todonotes} % .--- lista de tareas
\usepackage{lipsum}    %  --- genera lorem ipsum \ipsum[3] 3 párrafos

% ------ índices --------------------

\setcounter{tocdepth}{1}

%\usepackage[toc]{multitoc} % --- índice general en 2 columnas, por si se hace necesario

\usepackage{imakeidx}
\makeindex

% ------ ilustraciones --------------

\usepackage{graphicx}
\DeclareGraphicsRule{.tif}{png}{.png}{`convert #1 `dirname #1`/`basename #1 .tif`.png}


% ------ tipografía ---------------------

\usepackage{Alegreya} 			% --- U otra fuente que se desee
\addtokomafont{disposition}{\rmfamily} 	% --- Encabezados en Serif
\addtokomafont{labelinglabel}{\rmfamily} % --- Etiquetas en Serif
\addtokomafont{descriptionlabel}{\rmfamily\bfseries}
\setkomafont{caption}{\footnotesize\itshape} 		% --- Pies fuente pequeña
\setkomafont{captionlabel}{\usekomafont{caption}}	% --- Idém



\usepackage{microtype}  % --- Arreglos Tipográficos

\usepackage[autostyle,spanish=spanish]{csquotes} % --- mejora de citas

\usepackage{verse} % --- poemas

\newcommand*\varhrulefill[1][1pc]{\leavevmode\leaders\hrule height#1\hfill\kern0pt \vspace{1pc}}

\newcommand{\capitulo}[1]{\chapter{#1} \varhrulefill}

% ------ cuadros ---------------------

\usepackage{booktabs}
\usepackage{tabularx}
\usepackage{wrapfig}


% ------ idioma -----------------------

\usepackage[spanish,es-noindentfirst]{babel}	% --- en Español, por favor
						% --- primer párrafo de cada sección sin sangría
						
\usepackage[utf8]{inputenc} % --- codificación de caracteres (entrada)
\usepackage[T1]{fontenc}    % --- codificación de salida


% ------- referencias -----------------

\usepackage[nottoc]{tocbibind}
\usepackage{cleveref} 		% ---  Documentación http://tug.ctan.org/macros/latex2e/contrib/cleveref/cleveref.pdf


% ------- enlaces ----------------------

\usepackage[	 colorlinks=true, pdfstartview=FitV, linkcolor=blue, 
                 	citecolor=blue, urlcolor=blue, pdfborder={0 0 0}, 
			   	pdftitle={Sabia Vida}, pdfsubject=reflexiones,
				pdfkeywords={minimalismo,sabiduría},
				pdfauthor={Miguel de Luis}, pdflang={es ES}]{hyperref} %enlaces debe ir al final