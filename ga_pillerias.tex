\capitulo{Pillerías}

Los personajes jugadores están tocados por la suerte, especialmente
si son niños. Quizás sean sus ángeles de la guarda o quizás es que 
el mundo sea así y ya está. En cualquier caso cuando las cosas se
ponene realmente difíciles pueden contar con un depósito de suerte hasta
que su estrella se desvanezca.

Las buenas acciones, especialmente los actos heroicos, pueden hacer
recuperar el favor de los ángeles, haciendo que esa estrella especial
vuelva a resplandecer y recuperen o incluso ganen nuevas pillerías.

\section{Ser un pillo}

Si un personaje jugador quiere ser un pillo, lo anuncia al máster, que
lo aprobará siempre que no sea nada absurdo, desleal, o, en general, vaya
contra las reglas. El jugador \emph{paga al máster}\footnote{Recomiendo usar 
círculos coloreados para representa las pillerías, como los quese venden para 
jugar a las cartas.} el valor del favor, --que el máster habrá asignado--, en 
Pillerías, restando de su total. El máster narrarálo que suceda a continuación, 
dejando muy claro el papel de la suerte.

\begin{quotation}

El máster diría algo del estilo de \enquote{Afortunadamente la bala que te
disparó era defectuosa y se desvió} o incluso \enquote{Una paloma asustada se
cruzó delante del asesino justo cuando te iba a disparar}, o cualquier otra cosa
que fuera apropiado.

\end{quotation}

The following table details how many chits you can use for any such
favor. You aren't limited by the table; the players should feel free
to ask for any favor not listed in this table, as long as the game
master approves them... for a fair price in chits.

\begin{center}
\label{sub:Using Chits Table}
\par\end{center}

\noindent \begin{center}
\begin{tabular}{cc}
\toprule 
Chits Spent & Favor Won\tabularnewline
\midrule
\midrule 
1 & Something nice happens\tabularnewline
\midrule 
1 & Give me a clue\tabularnewline
\midrule 
1 & Ha! Missed me\tabularnewline
\midrule 
1 & Re-roll \tabularnewline
\midrule 
1 & Still breathing\tabularnewline
\midrule 
1 & Double or nothing\tabularnewline
\midrule 
1 & Let me try something cool\tabularnewline
\midrule 
1 or more & Found it!\tabularnewline
\midrule 
1 & I'm a hero\tabularnewline
\midrule 
2 & Each extra weak die\tabularnewline
\midrule 
2 & Just hurt\tabularnewline
\midrule 
3 & Something helpful happens\tabularnewline
\midrule 
3 & Each extra swell die\tabularnewline
\midrule 
3 & Triple or nothing!\tabularnewline
\midrule 
4 & Just a scratch\tabularnewline
\midrule 
5 & Can't touch me\tabularnewline
\end{tabular}
\par\end{center}


\subsection*{Explanations}


\subsubsection*{\emph{Something nice happens}}

For 1 chit, you can have an event happen that would require some help
in any given situation. For more substantial help you need to pay
3 chits, as discussed in \emph{Something helpful happens}. In any
case, this event must be plausible and make sense.
\begin{quote}
Examples of nice events include: A kind old lady buys you an apple
when you are hungry, or a small brave dog comes to your defense in
a fight. Perhaps you are seeking employment and a lawyer trusts you
to deliver some important documents to a client. If you do it well,
you'll have earned yourself a job. Maybe the common hospital room
is filled and you get your private room, just like the rich kids,
or perhaps you meet a friendly Civil War veteran just two days before
your American History exam at the night school.
\end{quote}

\subsubsection*{\emph{Give me a clue}}

For 1 chit you can ask the game master to give you a clue about some
secret or mystery, or some other element of the plot. The game master
can be as straightforward or enigmatic as the story requires. For
example: instead of just saying \textquotedblleft{}The butler did
it\textquotedblright{}, the game master could say something like \textquotedblleft{}The
master of politeness is stained in red\textquotedblright{}.


\subsubsection*{\emph{Ha! Missed me}}

For 1 chit you can dodge any attack delivered by a non-lethal weapon
or any unarmed attack (such as a kick). The attack is assumed to have
failed, so your character can act normally. 


\subsubsection*{\emph{Re-roll}}

Choose any dice you have already rolled and re-roll them at a cost
of 1 chit per die. You can re-roll all your dice for a total cost
of 3 chits.


\subsubsection*{\emph{Still breathing}}

When the game master announces your character is dead for any reason,
you can pay 1 chit to keep them alive. However, he is still gravely
injured and unable to stand up, so it could be wiser to pay 2 chits
for the ``Just Hurt'' favor. 


\subsubsection*{\emph{Double or nothing}}

Just as it says, pay 1 chit for the opportunity to double the result
of all your dice. The whole thing is a gamble that could turn against
you if you are unlucky again. See how it goes.
\begin{enumerate}
\item If after rolling your dice you don't like the results and want a chance
for \emph{Double or nothing}, you tell the game master.
\item He then would announce what is at stake. Usually you would lose 1
or 2 extra chits and fumble at your attempt. However, the game master
could say your character is hurt, or might fail spectacularly before
all your friends. If you agree with the consequences, you spend your
chit and roll one normal die. 
\item On an even result you can double your original throw. 
\item On an odd result however, you'll face the consequences that the game
master announced.\end{enumerate}
\begin{quote}
Let's say you are playing baseball. You rolled a mere 8 (according
to the game master, just enough to advance to first base). However,
the game master told you that a 15 would grant you a home run and
you desperately want that. So, you announce that you are taking your
chances at \emph{Double or nothing} and want to know what the consequence
of failure could be. The game master, perhaps in a mischievous mood,
announces that if you don't roll higher than your original throw,
your pants will fall down for the general merriment of your friends
and rivals watching the game. Plus, you'll have to pay 1 extra chit.
You agree to those terms and re-roll. Thankfully, this time you roll
a 4 (an even result), and double your original throw for a total of
8 x 2 = 16 chits; more than enough for a home run.
\end{quote}

\subsubsection*{\emph{Let me try something cool}}

Sometimes you just need to do something cool, like jumping on a horse
and then through a window to dodge a particularly nasty bully. The
problem is that the game master can assign an impossible target number
to such daredevil actions. In that case you can pay 1 chit to lower
that TN to 12.

\smallskip{}
The game master must agree that:
\begin{itemize}
\item a) Your action is cool enough (dangerous is not always cool); and
\item b) It has to be physically possible (even if unlikely) for a human.
\end{itemize}

\subsubsection*{\emph{Found it!}}

Whenever your character needs an item that's reasonable for him to
own or somehow find, you can pay 2 chits and your character will either
have it in his pockets or find it easily somewhere. The object must
not weigh more than five pounds or cost more than \$1 per chit%
\footnote{A 20-pound item will cost you 4 chits and 3 chits will buy you a \$3
coat.%
} spent. 


\subsubsection*{\emph{I'm a hero}}

For heroic actions in which your character, out of the goodness of
his heart and without motive for any reward, risks his life for the
sake of others or some noble ideal, you can pay 1 chit to lower the
TN to 10.\smallskip{}


As in \emph{Let me try something cool} the game master must agree
that: 
\begin{itemize}
\item The action is heroic enough; and
\item It has to be physically possible (however unlikely) for a human.
\end{itemize}

\subsubsection*{\emph{Each extra weak die}}

Before or after you roll the dice, 2 chits buy one extra weak die
to roll immediately before your throw. This extra die is lost as soon
as it is used. 


% \subsubsection*{\emph{Just hurt}}

% You can pay 3 chits to reduce a grave or dead damage result to hurt
% on the damage scale%
% \footnote{You will see more information about damage in Section \vref{sub:TDamage-Scale.}.%
% }. You could say the bullet did not touch a vital organ, the sword
% did not cut as deeply as it looked or that you fell on soft ground.
% In any case, you only received half as much damage.


\subsubsection*{\emph{Something helpful happens}}

For 3 chits, you can ask the game master for a helpful event to happen.
This event must be plausible without recourse to fantasy. It must
make sense, so restrain yourself as it cannot fully solve the situation
on its own. For example: if you are selling papers good weather would
definitely help, but you can't get someone to buy all your bundles
no matter how good your excuse is. In any case, the game master's
opinion is final on this matter.
\begin{quote}
Examples of helpful events include: A nurse on his way to the hospital
finds you hurt and decides to help; an angry wolfhound appears just
between you and the muggers who would rob you; the train you were
about to miss is delayed; the thundering rain which was going to doom
paper sales stops until you finish your last bundle; your court case
is assigned to a judge who has a soft spot for newsies, or while shopping
for new boots a new shoe shop opens and everything is selling at 25\%
off. 
\end{quote}

\subsubsection*{\emph{Each extra swell die}}

Before or after you roll, 3 chits buy one extra swell die. You must
roll your new die immediately and add the result to your throw. 


\subsubsection*{\emph{Triple or nothing!}}

Just as it says, pay 3 chits for the opportunity to triple all your
dice. The whole thing is a gamble that could be turned against you
if you are unlucky again. See how it goes.
\begin{enumerate}
\item If after rolling your dice you don't like the results and want a chance
for \emph{Triple or nothing!}, just tell the game master.
\item The game master then announces what is at stake. Usually you'd lose
3 to 5 extra chits and fumble at your attempt. Although, the game
master might say your character could be hurt, die or just fail spectacularly
before all your friends. If you agree with the proposed consequences,
you spend your 3 chits and roll one normal die. 
\item On an even result you can double your original throw. 
\item On an odd result however, you'll face the consequences that the game
master announced.
\end{enumerate}

\subsubsection*{\emph{Just a scratch}}

At the cost of four chits you can trade the effects of any injury
for a -1 malus%
\footnote{Refers to the level of sickness. You will see more information about
damage in Section 6.7.%
}. This is a pretty expensive option so use it with care. You can't
use it for any previous injuries or for accumulated damage; it must
be used immediately after your character has been hurt.


\subsubsection*{\emph{Can't touch me}}

For five chits your character can survive any single attack or accident,
suffering no effects whatsoever.


\subsection*{New favors}

The game master can grant any favors not included in this list at
a reasonable cost in chits. 


\subsubsection*{In short}

When your player needs a favor, refer to the favor table, pay the
chits and follow the rules. If the favor is not included in the table,
the game master may (or may not) concede at a reasonable cost in chits,
using the table as a guideline.


\subsection*{An example game session, continued}

So, we continue our adventure from the end of Chapter 3:
\begin{verse}
Martin: ``A spectacular success. Daniel, by sheer luck you spot er...
25 cents spread southwards in pennies.\textquotedbl{}

Tammy: ``Great, Daniel. Now, grab the money and go fetch him!\textquotedbl{}

James: ``I think we should make a torch first. Martin, is there anything
we can use to make a torch from the stuff around here?\textquotedbl{}

Martin: ``Hmm, that would be too much luck...\textquotedbl{}

James: ``OK, I'm pushing my luck here. Martin, what if I pay you
1 chit?\textquotedbl{}

Martin: ``Make it 3, and it's a deal.\textquotedbl{}

Tammy: ``Three? For one stupid torch?\textquotedbl{}

Martin: ``One stupid torch that you need now. A torch that could
be quite handy, you know.\textquotedbl{}

Daniel: ``I think 2 chits is fair for a torch.\textquotedbl{}

Martin: ``OK, it's 2 chits, but that's final. Take it or leave it.\textquotedbl{}

James: ``I take it. Here's my 2 chits.\textquotedbl{}

Martin: ``Great, thank you. You all start searching for materials
to make a torch but there's nothing you can see around here. Finally,
as you are about to give up, James steps on something and falls on
the ground, landing on his bottom.%
\footnote{Usually, James would have checked Agility to avoid such a fate, but
as this is just for fun, the players are letting the game master get
away with it.%
}
\end{verse}

\section{Earning chits}

The initial supply of chits might seem generous, but because they
can run out quickly, you'll be happy to know there are a few ways
of replacing them. Chits are not earned easily, as you will see in
the table below.

\begin{center}
\label{sub:Earning Chits Table}
\par\end{center}

\begin{longtable}{cc}
\hline
\toprule 
Event & Chits Earned\tabularnewline
\midrule
\hline
\endhead
\midrule 
Act of kindness & See rules\tabularnewline
\midrule 
Helping a rival & 2\tabularnewline
\midrule 
Helping an enemy & 3\tabularnewline
\midrule 
Heroic generosity & 2\tabularnewline
\midrule 
Outstanding honesty & 2\tabularnewline
\midrule 
Winning the adventure & See rules\tabularnewline
\midrule 
Winning a competition & 1\tabularnewline
\midrule 
Saved a life & 3\tabularnewline
\midrule 
Heroic feat & 2\tabularnewline
\midrule 
Major festivities & Varies\tabularnewline
\midrule 
A stranger helps you & 2\tabularnewline
\bottomrule
\end{longtable}
\begin{description}
\item [{Act~of~Kindness:}] As every newsboy knows, God helps those who
are kind and good. This might not agree with your opinion about real
life, but that's the way it works in this game. When a character performs
a random act of kindness towards somebody who is not a friend and
cannot easily answer in kind, roll a normal die. If you roll a 5 or
a 6, you earn 1 chit. The game master must be satisfied that the act
is appropriate to the adventure, is ethical and demands significant
effort from the character%
\footnote{Giving a flower to a homeless lady is nice, but will not grant you
the chance of getting a chit. Spending half an hour to find the parents
of a five-year-old girl who is lost in the city would.%
}. 
\item [{Helping~a~Rival:}] If you help somebody who is definitely not
a friend or rival in some kind of competition, you get 1 chit. Another
newsboy who makes fun of you because you sell less qualifies as a
rival. That straight-A student who tells the whole world every time
you get a C or worse, qualifies as a rival. A friend who sometimes
makes a joke, does not. When helping a rival you must not do anything
unethical or that would unlawfully harm anyone else. Again, the game
master must be satisfied that the act is appropriate to the adventure,
is ethical and demands significant effort from the character. 
\item [{Helping~an~Enemy:}] An enemy is the most serious form of rival.
Examples of enemies include a bully who is after you, a crooked cop,
somebody who has robbed you, and so on. Again, when helping a rival
you must not do anything unethical or that would unlawfully harm anyone
else. You don't get 3 chits just by helping a gang kidnap a pastor.
The game master must be satisfied that the act is appropriate to the
adventure, is ethical and demands significant effort from the character.
\item [{Heroic~Generosity~and~Outstanding~Honesty:}] This time the
game master must not only be satisfied that the act is appropriate
to the adventure, is ethical and demands significant effort from the
character; the action must be extraordinary. Ordinary honesty, like
not shortchanging or keeping your word does not grant you any chits.
Returning a wallet you found on the street with \$20 or confessing
a crime when nobody has a clue you did it, would indeed qualify. Likewise,
heroic generosity means an extraordinary effort either in money or
kind. For example: spending a full week volunteering at a soup kitchen
for eight hours a day, donating a full week's earnings or sharing
your food even when you are hungry.
\item [{Winning~the~Adventure:}] Once the game master is satisfied the
players have finished an adventure successfully, he will grant a number
of chits to the whole gaming group. It is the players' responsibility
to divide the whole lot as they believe is fair. Some groups may simply
divide the points evenly, while others may choose to reward the best
players, or the one who brought the pretzels and soda. The exact number
of chits will vary from adventure to adventure. Generally, the harder
the adventure is the more chits the group will receive.
\item [{Saved~a~Life:}] When a player character makes a significant effort
by taking an action that decisively helps rescue somebody from certain
danger, he earns 3 chits.
\item [{Heroic~Feat:}] Two chits are awarded to any heroic feat that does
not lead to saving a life.
\item [{Major~Festivities:}] On specific days such as Christmas, there
is so much fun and goodwill that your character recovers his energies
and optimism. For details refer to Chapter 8. \ref{sec:Seasons}
\item [{A~Stranger~Helps~You:}] If a non-player character who does not
know your character helps you when you are in some great trouble,
you receive 2 chits. The game master should use this option only when
a player is in dire straits, not to reward careless playing.\end{description}
\begin{quote}
\emph{\textquotedbl{}In fact, if it had not been for a good-hearted
turnpike-man, and a benevolent old lady, Oliver's troubles would have
been shortened by the very same process which had put an end to his
mother's; in other words, he would most assuredly have fallen dead
upon the king's highway. But the turnpike-man gave him a meal of bread
and cheese; and the old lady, who had a shipwrecked grandson wandering
barefooted in some distant part of the earth, took pity upon the poor
orphan, and gave him what little she could afford - and more - with
such kind and gentle words, and such tears of sympathy and compassion,
that they sank deeper into Oliver's soul, than all the sufferings
he had ever undergone.'' Oliver Twist, Chapter VIII}
\end{quote}

\section{Maximum number of chits}

Player characters can save the chits they accumulate during their
adventure as a good luck reserve for times of real trouble. However,
there is a limit to the number of chits they can accumulate. Chits
won by a character who has reached this limit are wasted and cannot
be used or shared.

\begin{center}
\label{sub:Maximum Number of Chits Table}
\par\end{center}

\begin{longtable}{cc}
\hline
\toprule 
Age & Can have up to\tabularnewline
\midrule
\hline
\endhead
\midrule 
9 & 22 chits\tabularnewline
\midrule 
10 & 20 chits\tabularnewline
\midrule 
11 & 18 chits\tabularnewline
\midrule 
12 & 16 chits\tabularnewline
\midrule 
13 & 14 chits\tabularnewline
\midrule 
14 & 12 chits\tabularnewline
\midrule 
15 or more & 10 chits\tabularnewline
\bottomrule
\end{longtable}


\section{Sharing chits}

Players can transfer chits to other player characters. For every 3
chits you share, the other character receives 2. If you share 2 chits
the other character would receive 1 chit. As you can see, sharing
3 chits gets a better rate. This is quite a good number of chits to
share, so reserve them for an emergency.

Sharing chits does not qualify for either an act of kindness or heroic
generosity, so you do not earn chits by sharing them. However, you
can consider yourself a good friend, tap yourself on the back and
smile proudly.


\section{Losing chits}

There is a difference between spending and wasting chits. A player
can waste his chits for re-rolls that were not worth it or he can
save them for when they are sorely needed. However, there are actions
and situations that will cause you to lose chits. These are introduced
for game balance and to help players role play better and think deeper.

There are three kinds of situations that will eat your chits up. First
of all, losing at something you are \emph{good at} or in which you
have given significant effort; second, acting against the Newsies
Code%
\footnote{See chapter \vref{cha:The-Newsboy-Code}.%
} and third, significant emotional distress. Some of these situations
are detailed in the table below. As always, the game master should
use discretion and consider this table as a guide, not the Bible.

\noindent \begin{center}
\begin{tabular}{cc}
\toprule 
Event & Chits Lost\tabularnewline
\midrule
\midrule 
Fumbling at something you are \emph{good at} & 1\tabularnewline
\midrule 
Fumbling after using 1 or more chits & 1\tabularnewline
\midrule 
First time your character steals & 3\tabularnewline
\midrule 
First time your character begs & 1\tabularnewline
\midrule 
Loses at some competition & 1\tabularnewline
\midrule 
Cowardice, if friends are in danger & 2\tabularnewline
\midrule 
Going hungry, per day & 1 or 2\tabularnewline
\midrule 
E-grade clothes in cold weather & 1\tabularnewline
\midrule 
D or E-grade clothes in freezing weather & 2\tabularnewline
\midrule 
Homeless, per week & 1\tabularnewline
\bottomrule
\end{tabular}
\par\end{center}
\begin{description}
\item [{Stealing:}] Caught or not, your character loses 3 chits the first
time he steals.
\item [{Begging:}] When your character has to beg for money from strangers
(the very first time), your character loses 1 chit for the humiliation.
\item [{Going~Hungry:}] Every night you go to bed without having been
able to eat at least one full meal, you lose 1 chit; 2 if the weather
is cold.
\item [{E-Grade~Clothes~in~Cold~Weather:}] E-grade clothes are not
warm enough to protect you in cold weather. For every full day that
you have to face wearing E-grade clothes in cold weather, you lose
1 chit%
\footnote{More rules about weather in Chapter \vref{cha:Optional}.%
}.
\item [{D~or~E-Grade~Clothes~in~Freezing~Weather:}] Freezing weather
is severe enough that only C-grade clothes or better can protect you.
For every full day that your character has to face wearing nothing
better than D or E-grade clothes in freezing weather, your character
loses 1 chit.\end{description}
\begin{quote}
You cannot lose more than 5 chits in any given season due to hunger,
foul weather or homelessness. The game master may choose to ignore
the loss of chits between adventures.
\end{quote}

\section{No chits? Big problem.}

Having no chits means more than just not being able to re-roll or
buy favors from the game master. Your character believes he's out
of luck and feeling deeply sad. Call it a depression if you like,
but I would not get that pretentious. Why is being so sad a problem?
Because until you get at least 1 chit (or somebody shares one with
you), every time you attempt an action that demands a dice roll you
have to first roll one normal die in the following table.

\begin{longtable}{cc}
\hline
\toprule 
Die (nd) & Result\tabularnewline
\midrule
\hline
\endhead
\midrule 
1 & ``It's no use trying.'' (Make no attempt.)\tabularnewline
\midrule 
2-3 & ``I'll just fail.'' (TN + 1)\tabularnewline
\midrule 
4-5 & No effect. (Roll normally.)\tabularnewline
\midrule 
6  & ``An angel smiles.'' (Gain 1 chit and roll normally.)\tabularnewline
\bottomrule
\end{longtable}


\subsubsection*{Exceptions}
\begin{enumerate}
\item You don't roll on this table while in a serious fight or during an
emergency. Adrenaline and instincts won't let you die without a fight.
\item Non-player characters don't have chits and they won't use this table. 
\end{enumerate}

\subsubsection*{Explanation of results}
\begin{itemize}
\item ``I'll just fail.'' Your character is so convinced of being a failure
that he is unable to make a full effort. Raise the TN by 1; better
luck next time.
\item ``An angel smiles.'' Call it God's help, endorphins, or whatever
you like. Your character has just recovered from his sadness, as young
persons often do. You get 1 chit; use it with care.
\item ``It's no use trying.'' Instead of trying, your character blames
his bad luck on the world, his parents or his friends. He starts whining,
rolls himself up in ball or does anything but what he intended to
do. He can try again once he is over it, after five minutes has passed.
\item ``No effect.'' Just as it says; you roll your dice normally. \vfill{}

\end{itemize}
\begin{center}
\vfill{}

\par\end{center}
