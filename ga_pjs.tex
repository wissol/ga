\chapter{Personajes Jugadores}

¿Crean los personajes las grandes historias? ¿O es la aventura quién hace a los héroes? Ni idea, lo que sí sé es que si empiezas por unos buenos personajes, añades una complicación y un mundo y tienes una aventura. 

En la mayoría de esos juegos de rol que, por prudencia, vamos a llamar vulgares y corrientes para heroínas sin pelo en el pecho, los personajes jugadores son una especie insípida con conocimientos arcanos, poderes mágicos y fuerza sobrehumana. ¡Bah! Demasiado fácil. En Pillastres las cosas son un poquito más interesantes.

Tu personaje comienza siendo un poco, como lo diría, un poco niño. O, con un poquito de suerte, adolescente. Ah, y pobres, y, ya puestos, sólo para redondearlo un poco, huérfanos.

Este juega describe a los personajes mediante números que te dicen, por ejemplo, cuántos dados debes tirar cuando tu personaje intenta huir de un fantasma, o qué pasa si te caes por las escaleras; lo fuerte que eres o las cosas que sabes hacer.

\minisec{Creación de Personajes}

Este capítulo te enseña a crear tu propio personaje. Por ahora échale un vistazo para que tengas una idea de las opciones que tienes. Después vuelve al principio de este capítulo y ve creando tu personaje. No te preocupes si no te queda perfecto, o si tienes un poco de mala suerte; lo importante del personaje lo pondrás tú, con tu imaginación.

Necesitarás un par de dados normales, papel y lápiz; al final de este libro tienes una hoja de personaje que puedes imprimir o copiar. La hoja de personaje es el mejor sitio para guardar toda la información de tu personaje 

\section{¿Quién eres?}

\subsection*{Edad}

La edad de tu personaje es muy importante. Tira un dado normal. Anota el resultado y suma 8: esa es la edad de tu personaje. Por ejemplo, si sacaste un cuatro tu personaje tendrá doce años; porque 4+8 = 12. A poco que pienses te darás cuenta de que los personajes de ¡Pillastres! comienzan el juego con edades que van desde nueve (1+8) a catorce (6+8).

\minisec{Excepción}
Si tienes menos de 16 años --me refiero a ti, el jugador, no a tu personaje-- no necesitas tirar el dado; si lo prefieres, puedes sencillamente escoger la edad del personaje que quieras, siempre que sea entre nueve y catorce años.

\minisec{Ventajas y Desventajas}
En este juego cada edad tiene ventajas y desventajas. Los mayores son más fuertes y saben más cosas, es cierto, pero los jóvenes tienen más suerte; en forma de más pillerías. Las pillerías son una serie de ventajas que tiene tu personaje por tener más suerte y que les servirá para escapar de más de un apuro, pero ya hablaremos de ellas más tarde.

\minisec{En resumen}

Tira un dado y añádele 8. El resultado es la edad de tu personaje.

Si tienes menos de 16 años, puedes escoger la edad de tu personaje siempre que sea entre 9 y 14 años.

\subsection*{Nombre y género}

Tu personaje debe tener un nombre y un apellido. Escoge el nombre que quieras y anótalo en tu hoja de personaje. 

Escoge también el género de tu personaje, el que tu quieras, no es obligatorio que tu personaje tenga tu mismo sexo, ni que se corresponda con su apariencia.

\minisec{Cumpleaños}
También necesitamos saber el día del cumpleaños de tu personaje. No incluyas el año, solo el día y el mes; las partidas de ¡Pillastres! transcurren en “aquellos tiempos, cuando los hombres eran hombres” y no en ningún año en concreto.

\section{Atributos}

\subsection{Definiendo los atributos}

Si le echas un vistazo a tu hoja de personaje verás claramente cuatro valores: \textsc{Fuerza, Salud, Agilidad, Educación, Ojos} y \textsc{Oídos y Carisma}. Estos datos son muy importantes porque definen las capacidades básicas de los personajes.

\begin{description}
\item{\textsc{Fuerza:}}
Mide lo fuerte que eres. Usamos el atributo de Fuerza para comprobar si puedes levantar algo muy pesado, romper cosas o defenderte de algún bruto.

\item{\textsc{Salud:}}
Mide la resistencia de tu personaje a las enfermedades, la fatiga, los venenos y los tartazos con tartas de crema. Lo de los tartazos no es broma.

\item{\textsc{Agilidad:}}
Nos indica la facilidad y gracia de los movimientos de tu personaje; sirve para saltar, hacer atletismo, montar una bicicleta, correr o esconderte de una banda de maleantes.

\item{\textsc{Educación:}}
Mide todo lo que tu personaje sabe del mundo. El atributo de Educación rige todo lo que puede aprenderse y conocerse; aunque no sea en un colegio. Sirve para conocer datos históricos y geográficos, desde luego, pero también para hacer reparaciones mecánicas o saber cómo vendar una herida.

\item{\textsc{Ojos y Oídos:}}
Mide la capacidad que tiene tu personaje de entender lo que ocurre a su alrededor. Aunque lo llamemos Ojos y Oídos, en realidad incluye los cinco sentidos de tu personaje; lo que pasa es que Ojos y Oídos son los órganos que normalmente se usan más en las partidas, cuando intentas buscar un mensaje secreto escondido entre los cascotes de unas ruinas o discernir si los pasos que escuchan a tu espalda provienen de un gracioso gato o un taimado espía.

\item{Carisma:}
Este atributo mide en un solo valor lo mono, simpático, agradable, guapo, alegre y atractivo que es tu personaje. El atributo de Carisma te vendrá muy bien para vender periódicos, hablar un idioma extranjero, convencer a un caníbal de que sabes muy mal, mendigar, tocar un instrumento musical, cantar o actuar tan bien que la gente te pague por ello.

\end{description}

Los atributos se miden en dados, que van a ser los dados que lances cuando necesites saber si tu personaje consigue o no lo que se propone. 

Ejemplo: Pepe tiene 3 dn (dados normales) en Carisma y quiere comprarse una bicicleta vieja que cuesta \$20, pero sólo tiene \$15. Intenta convencer al dueño de que le haga una rebaja usando su Carisma. El Director de Juego le dice que vale, pero que tiene que sacar un 15 con los dados, que es bastante difícil. Pepe lo intenta y saca un 4, un 3 y un 2, por un total de 9. ¡Mala suerte! Ha sacado menos y el dueño no acepta hacerle ninguna rebaja.

Cada atributo tiene que tener asignado al menos un dado; sólo algunos monstruos carecen de algún atributo --por ejemplo los zombies no tienen Carisma de feos y tontos que son-- y entre más y mejores dados tengas mejor será tu personaje en ese atributo.

\section{Generando tus atributos}

Tus atributos necesitan dados, ¿cómo los consigues? Pagándolos con nuestros maravillosos Puntos de Atributos ó PA para los amigos. Empiezas con tantos Puntos de Atributo como tu edad más siete. O sea:

PA = Edad + 7

Con estos Puntos de Atributo puedes “comprar” dados para los atributos de tu personaje. Cada dado débil cuesta 1 PA, y los dados normales cuestan 2 PA. También puedes convertir un dado débil en un dado normal por 1 PA.

Todos los atributos tienen que tener al menos un dado, ya sea fuerte o débil. En cada atributo puedes comprar hasta cuatro dados, ya sean fuertes o débiles.

Como los dados normales no dan el doble de puntuación que los dados débiles pero cuestan el doble, es mejor que compres cuatro dedos débiles en un atributo y luego, si te quedan puntos, conviertas alguno de esos dados débiles en dados normales.

Por si te lo preguntabas, no puedes comprar dados malos ni dados fuertes cuando creas tu personaje; ya verás como se usan estos dados cuando lleguemos al capítulo que explica el sistema de juego.

\section{Paso}

Paso es un número auxiliar que mide el número de metros que tu personaje se mueve por turno cuando camina y depende de tu edad. Los menores de nueve años tienen un paso de 2 metros por turno; de nueve a once son tres metros por turno, de 12 a 14 son cuatro metros por turno, y los mayores de catorce andan a cinco metros por turno.

Sí, puedes moverte más rápido si corres o usas un vehículo, pero dejaremos esos detalles para el capítulo de “Peleas y Persecuciones”. Por ahora, todo lo que tienes que hacer es anotar el Paso de tu personaje en la hoja de personaje.

Obviously, your character can move faster by running or sprinting, but we will leave those details to Chapter 6 “Fights and Chases”. Right now, all you have to do is record your character's pace on the character sheet.

\section{Habilidades}

Las habilidades son parecidas a los atributos

3.4 Skills

These are your skills

Skills are stats just like attributes, only less broad. Agility (attribute), tells us how well you move in general, while Riding (skill), for example, tells us how good you are at handling a bike.

This game has 21 skills open to player characters:

Fisticuffs (STR), Throwing (STR), Athletics (AGI), Climbing (AGI), Dodging (AGI), Locks (AGI), Pickpockets (AGI), Riding (AGI), Stealth (AGI), Swimming (AGI), Vehicles (AGI), Academics (EDU), Crafts (EDU), Healing (EDU), Mechanics (EDU), Streetwise (EDU), Observation (EE), Shooting (EE), Languages (CHA), Performance (CHA) and Sweet Talk (CHA) 

You will figure out that Athletics is good for climbing, jumping and playing football; Fisticuffs comes in handy for brawling and Healing helps to fix people after that brawling. Other skills might be a little harder to figure out. If you need help, they are better explained in Chapter 3, Section 3.8. [cha:Testing-skills-and]

Every skill is linked to an attribute that is shown in brackets. For example, Academics (EDU) is a skill linked to the Education attribute. That means if you are good at Academics (an Education skill), you roll using your Education dice, but upgraded.

Notice that no skill is linked to the Health attribute.

How you generate your skills

First, find out how many skills your character is good at. These are as many as his age minus 6. If your character is nine, he's good at 9 - 6 = 3 skills. Write them down on your character sheet or a scrap of paper under “good at”.

Then, find out how many skills your character is bad at. These are 18 minus his age. If your character is 12, he's bad at 18 - 12 = 6 skills. Record these under “bad at”.

Those skills you have not chosen to be either good at or bad at, you are OK at. We will deal with the exact meaning of these in Chapter 3, Section 3.2.1. Right now, all you need to know is that being bad at a skill decreases your chances at completing any tasks associated with that skill because it downgrades your dice. Being good at a skill upgrades your dice.

As you can see, the older your character is the more skills they will be good at and the less skills they will be bad at. That makes perfect sense, though it makes younger characters weaker. In the next section titled “Chits”, you will find a system that somehow compensates for that, especially if the player is smart and creative.

In short

• A player character is good at as many skills as his age minus 6. 

• A player character is bad at as many skills as 18 minus his age.

• The player chooses freely the skills he is good at or bad at. He records these on the character sheet.

• Any skill not chosen as either good at or bad at is an OK at skill. The player may record these on the character sheet for reference.

Choose wisely. If your character is bad at Fisticuffs, he'd better be good at Athletics. Did you notice the attributes' abbreviations in brackets next to the skill name? These are important. I will be telling you more, but just so you know, being good at Riding, a skill of Agility, makes your Agility dice better when you are riding bikes, while being bad at Riding, will make your Agility dice worse. So, if you have a lot of dice in Agility, being good at Riding will make you a fantastic racer, but if you have few and bad dice in Agility, you'll only be average.

Can you read?

If your character is bad at Academics he can't read. If your character is good at Academics he can read perfectly.

If your character is OK at Academics, he can read, but might misunderstand a few things. The game master may ask you to check your Academics skill any time you try to read something more difficult than a children's book. 

3.5 Chits

What are chits?

Chits are points a player can use to buy favors from the game master, such as rolling dice again or making dice stronger. Chits can also be used to avoid death or change the flow of the story, at an increased cost. Don't be too happy spending them, because chits are hard to refill. If you run out of chits your character could become a crybaby who would rather whine than try hard at anything. 

I will be sharing more about chits later in Chapter 5, [cha:Chits] but for now know that the more you have the better.

How many chits you begin with

You begin with as many chits as 20 minus your character's age. If your character is 12, you have 8 chits; if your character is nine you have 11 chits, and so on. 

During the game, you will be able to earn more chits and share them with other players. The maximum number of chits you can have at any time is twice as much the starting chits of your character's age. This maximum does change as your character grows older in the game. If you begin at nine years of age your maximum number of chits will decrease from 22 to 20 (twice the starting chits for a 10-year-old).

In short

• Use chits to buy favors from the game master.

• Every player character begins with 20 minus age chits (20 - Age).

• Player characters can have up to as many chits as twice the starting chits for their character's age.

• Only player characters can have chits.

3.6 Equipment

The characters of this game are poor orphans and they have to earn the money for their own bread and butter. Even so, they begin the game with a few items and some money.

Dress up or fade away

First, let's see how well-clothed you are. Roll two normal dice and check the number you get in the following table. 

     Clothes Table      



          



What do these letters A to E mean? A = best quality, while E = worst. Leave it to your imagination to provide the details, however, some suggestions follow:

E-Grade Clothes Kit: If your character is wearing E-grade clothes, he looks just like Oliver Twist did right after arriving in London. He owns a shirt, knee-long breeches and his underwear; that's that. He is barefoot, which could be nasty in winterThat aside, being barefoot is not much of an issue. It's normal for children to go barefoot in New Paris City; even outdoors. But don't expect to be admitted to any fashionable place without decent shoes, and don't even think about selling some of the most expensive papers, you street rat. . Oh, and don't get me started about his undies. 

D-Grade Clothes Kit: If your character is wearing D-grade clothes he looks a bit better, but not much. He has everything the E kit has, plus he gets a worn-out jacket and a newsboy cap, and perhaps some ragged boots or shoes, reserved for foul weather. Besides all that, I'm afraid that probably some piece of his clothing is two sizes larger or smaller than you want.

C-Grade Clothes Kit: If your character is wearing C-grade clothes, he looks like the average child of a working family on his way to the factory or, if he's lucky enough, to school. If a boy, he wears a shirt, his pantsLong or short is your choice, but mind the weather., a jacket, a newsboy cap, a pair of shoes and socks. A girl would own a simple dress with long stockings, plain shoes and some headgear. Everything is of plain color, but reasonably clean and well made. You will not need to worry about winter with these clothes; it could still feel cold, though. Characters wearing C-grade clothes and worse will only own two sets of underwear, so they need to wash these thoroughly every night... or pretend to do so.

B-Grade Clothes Kit: This grade of clothes is a little better. It includes everything a C-grade clothes kit has, plus extras such as a waistcoat, a cheap handkerchief with your initials sewn on it and a scarf for winter. You also have plenty of underwear, so you don't need to get your undies clean every night. A B-grade clothes kit also includes an old C-grade clothes kit.

A-Grade Clothes Kit: It is a rare sight to see a newsboy in this rich-boy outfit. It includes everything B-grade clothes have, but items are much better made and with brighter colors. Wearing this outfit is how you want to sell The Hawk or some other fancy newspaper in Benjamin Franklin's park. In A-grade clothes you pretty much look like Little Lord Fauntleroy, which is generally best anywhere but in dark alleys. An A-grade clothes kit also includes an old C-grade clothes kit.

Wealth

Your character will want to buy things, and the best way to know if he can afford something is to determine how much money he has. I will be sharing the exact details in Chapter 4. Right now, all you need to do is roll a couple of dice, add your character's age and multiply the total by 10. The result is how many cents your character begins with.

In short

• Starting wealth = ( 2 dice + age ) x 10 cents

Example: Anne is creating Tom, a newsboy, who is 12. She rolls two dice, getting a 7. Now, 7 + 12 = 19, which means Tom begins with 19 x 10 = 190 cents or $1.90.

Other useful stuff

There is nothing as full of wonder as the pockets of a child. Well, to be completely honest, full of wonder and plain, old-fashioned garbage. Instead of buying or choosing your equipment before the game, you must roll in the Random Stuff Table. If the dice show their uglier sides to you, remember that even the most seemingly insignificant item can be gold in the hands of a lively mind. 

Roll two normal dice, but this time don't add them. Instead, choose one to be the first die and the other to be the second die. Check the table, knowing that the first die will be the first number and the second die will be the second number. If you rolled a 1 with your first die and a 4 with your second die, check the “1-4” box in the following table; you get a notebook. If you rolled a 3 and a 2 with your dice, the “3-2” box will provide you with some yummy sausages.

You must roll at least once, and up to seven times on the Random Stuff Table. Be careful, because if you roll a double six (6-6) you lose everything, including your money, and you will not be able to roll again. 

     Random Stuff Table      



Notes

1. Roll two normal dice to determine how many sausages you get.

2. NPET stands for New Paris Elevated Train company, the main New Paris transportation system. Each NPET token is good for one-way travel to anywhere in New Paris. 

3. Good for four letters addressed anywhere within the state of New Paris.

4. Old and rusty, but working. 

5. A small cute and harmless animal.

6. You've been robbed and lost everything you got on the Random Stuff Table. You lose all your money and your clothes rating drops to E. You can't make any further roll on this table. Tough luck!

In short

• Roll for your clothes.

• Roll for your money. You begin with (2 dice + age ) x 10 cents.

• Roll at least once and up to seven times in the Fabulous Random Stuff Table of 36 Miscellaneous Stuffous.

• If you roll a double 6 you lose everything you acquired in the table.

3.7 Background

How did your character become a newsboy living on his own on the streets of New Paris City? Everybody has a story. An interesting story can help a poor orphan with a good imagination sell some newspapers. Beyond that, knowing where you come from will make your character worth more than a bundle of numbers and stats.

Are you really an orphan? If your parents are alive, then what about them? Is your father in prison for life? You might want to keep that to yourself, though. Is your mother locked in an asylum for her own safety? Are they so poor the best thing they could do for you was to buy you a ticket for New Paris City and pray? What became of your brothers? Perhaps you could be the sibling of another character; just ask another player if you think that would be cool. Were you born in America or are you an immigrant? Were you born in New Paris City or did you, as many others, travel there for the opportunity to make your dreams come true? What are those dreams, by the way? Fancy to be a lawyer, perhaps the president despite all odds, or do you want to try your chances at baseball?

Add all the details you want and write them down on your character sheet. If you can only think of a couple, that's alright. You can always add to your character's story as you play the game. But please, do speak with your game master to check that your family story fits the adventure. He might be thinking of a story about kids who just arrived from Europe, or about the children of a scientist murdered by a nefarious rival or...

If you aren't that sure where your character is from you may want to use the Background Table that follows. Roll three normal dice, thrice. The first roll will determine where you are from. The second and third will see what happened to your mother and father.

Background Table



          



Roll once for place of birth and once for each parent.

Dead: Your father or mother is dead and unless the information is wrong, there is nothing anybody can do about it.

Prison: Your father or mother is serving a very long sentence in prison, possibly for life. You can only visit once a year and he or she doesn't write you much. Yeah, it's kind of sad.

Hospital: Your father or mother suffers from a long-term sickness that keeps him or her in the hospital or some other facility. If it's not a transmissible or mental illness, you could be allowed to visit once a week.

Missing: You don't know where your father or mother is. Perhaps they did not come home from work one day, or their ship sank and the body was never found. Or, maybe your parent had to run away from police, criminals or creditors.

Unknown: You have no idea who your father or mother is. If you know your father but not your mother, consider that your father was simply somebody who cared for you, because he had a kind heart, or perhaps to send you begging or to pick a pocket or two, just as Fagin did with Oliver Twist. If both your parents are unknown, then you spent your first years in an orphanage until you (or somebody else) decided you were old enough to care for yourself.

Adventure: You aren't quite sure; maybe it's a story you made up for yourself when you were very little. Perhaps somebody told you just to comfort you, or you believe your parent is doing something special. It could be that you are the secret son of a prince of some small nation in central Europe, or that your mother is an elf or is working in a mission somewhere far, far away.

In New Paris, as in real life, many newsies, bootblacks and young peddlers live with their families. Player characters, however, are all on their own. Parents (uncles, grandparents and any adult family member) are either dead, unknown, far away, locked in prison or unable to help or make contact with the player character. In practical terms, every player character in this game is an orphan.

In short

• Figure out your character's story.

• Remember that your character's parents and any other grownup members of his family must not be able to help. They could be dead, sick, in prison, lost in the jungle...

• If you want, you can roll on the Background Table.

3.8 The pack

Newsies  Bootblacks assumes all player characters are friends who care for each other. So, it is a good idea for players to coordinate themselves when they are designing their characters. Pay attention to what each of you is going to be good at or bad at. Ask the game master for hints about what could be needed in the game. If the game master says the adventure is all about spies, then you'd probably want to be good at Stealth or Languages. Think what could happen if everybody was bad at Streetwise.

Also, you should have a story explaining why you are all friends. Some of you could be siblings or cousins. Perhaps you all arrived in New Paris on the same ship from Europe and your parents died on the journey. Usually, you just met in the streets, got along well and decided to be friends. Or, there could be other reasons. Perhaps the game master wants you to meet in the first scene you play in the adventure. Be sure to ask your game master first.

In short

• Ensure your characters make a good team.

• Make up your own pack story.

• That's that!

Now that you have your characters ready to go sell some papers and save the world at the same time, the question is, are you up for the challenge?

Character creation: an example

Let me guide you as I create a character using the rules I've just outlined in this chapter. First of all, I roll one single die to determine my character's age. The result is a 4, so my character's age is 4+8 = 12. I decide my character is a boy and his name is Allan Krebs.

Then I need to buy dice to choose his attributes (age + 7). He's 12, so I have 12 + 7 = 19 attribute points (AP) to buy dice. I decide to buy only weak dice (wd), as they give more bang for the buck. I buy three wd for Strength, two wd for Health (so Allan is a little weak), three wd go to Agility, four wd to Education (I want Allan to be smart), three wd for Eyes  Ears, and the last four wd go to Charisma. No attribute has more than four dice and each one has at least one, as the rules command.

His attributes are now finished:

Strength = three wd, Health = two wd, Agility = three wd, Education = four wd, Eyes  Ears = three wd, Charisma = four wd

Then I decide which skills Allan is good at and bad at. As he is 12, he's good at 12 – 6 = 6 skills. I choose Academics, Sweet Talk, Observation, Streetwise, Athletics and Shooting. He'll be bad at 18 – 12 = 6 skills. I choose Pickpockets, Locks (he's no thief), Vehicles, Mechanics, Crafts, (he's not good at fixing stuff) and Riding.

He'll be OK at the remaining nine skills. These are Climbing, Dodging, Fisticuffs, Healing, Performance, Languages, Stealth, Swimming and Throwing. I don't need to write these on the character sheet, but I'm going to do it anyway to speed up the game.

Allan is good at Academics, and that means he can read well enough to not need to check for his Academics skill when trying to read any text in English. 

Allan is 12 and that means he will begin with 20 - 12 = 8 chits. His maximum number of chits is twice this number, or 16.

Let's see about his stuff now. I roll two dice for his clothing, using the Clothes Table and argh, I got an 8!; that means Allan's clothes are of the lowly D grade. He's barefootHe's saving up his shoes for winter. and his clothes could use a little mending and cleaning, I'm afraid. 

Money, that's what I want. Allan is 12 and I roll an 8 on two dice so I get (12 + 8) x 10 cents = 20 x 10 cents = 200 cents or $ 2.00.

I decide to try my luck with the Random Stuff Table, rolling seven times (the maximum). On my first roll I get a 3 and 5, a 3-5 which means six candles. I make another six-dice throw and get six more candles, a pack of expensive sweets; let's make them Turkish Delights (I confess myself a Narnian in spirit), a harmonica (that's cool) a notebook and a bowl; a curious mix.

Finally, I see to my character's background. I'm not sure of where I want him to be from, so I roll using the Background Table as a guide. I roll an 11, so he's from the Inner Expanses. Then, I roll again to determine his father's and mother's background, getting a 12 and a 13; they are both missing.

I note everything down on the character sheet and Allan is now ready for a great adventure in New Paris. 

3.9 Special cases

Hotel staff

Maids, hotel boys, apprentice cooks and other hotel staff have much less freedom than a newsboy. They work all day, live at the premises and take orders from everybody else. In exchange, they get a uniform, a weekly wage, decent meals and a bed in a shared room. If they work hard and do well enough, they could raise in the ranks and become a hotel manager. These jobs are not easy to get.

Hotel staff characters are created using the same rules as Newsies, with the following differences:

1. They get a B-grade uniform; those working for a luxury hotel get an A-grade hotel boy or maid uniform. They can keep it, even if robbed, as the hotel will replace it for them.

2. They cannot own animals of any kind.

3. They get an extra die to determine their starting wealth.

As hotel staff characters are restricted to their hotels, any player wishing to play them must first seek permission from the game master. A game master might also want to run an adventure exclusively with hotel staff characters.

Bootblacks

Bootblacks are those who make a living out of cleaning and waxing boots and shoes. Many among them are also newsboys, providing both services for some extra benefit and a lot of extra trouble. Bootblacks are created just like newsboys, but they must somehow acquire a Bootblack box. The Crafts skill is also a must for bootblacks and most trades.

Bootblacks are allowed in any gameThat is, unless the game master or game group is thinking of something very special., and always welcomed in the Newsboy Lodge.

Messengers

Messengers work for a telegraph company delivering cables from the telegraph stations to clients anywhere in New Paris. Messenger boys must be physically fit, able to read, own a bicycle (the company isn't providing one, sorry) and be at least OK at Streetwise. Messenger boys trade the freedom of being a newsboy for weekly wages and a C-grade uniform.

The game master may not allow messengers in the game, so check with your game master first.

Errand kids

Errand kids work for companies, hotels, institutions or professionals, doing errands and whatever else they are asked. While many errand kids work for a single employer, player characters should most often be of the independent kind, working for young lawyers and others who cannot hire permanent assistants. While anybody can be an errand girl or boy, good Charisma, being good at Streetwise and Crafts, owning a bicycle and having respectable clothing and manners do help.

Independent errand kids are always allowed in any game and welcomed in the Newsboy Lodge. Mind that many newsies, bootblacks and other kinds of street kids work as errand kids to supplement their income.


Apéndice: Fuentes de Inspiración
1. Oliver Twist
2. August Rush
3. Les Miserables
4. Alabama Moon
5. Jules Verne
6. Castle Falkenstein