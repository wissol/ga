\chapter{Creación de Personajes}
\varhrulefill

\section{¿Quién eres?}

\subsection{Edad}

La edad\index{edad} de tu personaje es muy importante, porque la edad da mejores oportunidades para aprender y desarrollarse. En general, un personaje adulto y joven será más fuerte que uno de edad media o, de un niño. Sin embargo, los niños tienen en este juego un secreto, una fuerza especial de la fortuna que hace que, de maneras misteriosas, consigan triunfar donde sería imposible. 

Puedes elegir la edad de tu personaje como gustes, entre 9 y 69 años. Los personajes más jóvenes todavía son demasiado pequeños para irse de aventuras y para los mayores su idea de aventuras es saltar a la comba con sus nietos. Si no estás seguro, también puedes lanzar tres dados, y dejarlo a la suerte. Solo tienes que sumar los dados y consultar el resultado que da el cuadro siguiente que, como verás, favorece los personajes más jóvenes.

\begin{table}[h]
\footnotesize
\begin{tabular}{lcccccccccccccccc}
\toprule
Dados&3&4&5&6&7&8&9&10&11&12&13&14&15&16&17&18\\\midrule
Edad&9&10&11&12&13&14&15&16&17&18&19&20&21&22&23&24\\\midrule
\bottomrule
\end{tabular}
\caption{Edad Aleatoria}
\end{table}

\minisec{Cumpleaños}
También necesitamos saber el día del cumpleaños\index{cumpleaños} de tu personaje. No incluyas el año, solo el día y el mes; las partidas de \textsc{Grandes Aventuras} transcurren en \enquote{aquellos tiempos, cuando el mundo rebosaba de oportunidades} y no en ningún año en concreto.

\subsection{Nombre y género}

Tu personaje debe tener un nombre\index{nombre} y, si quieres, uno o varios apellidos. Escoge el nombre que quieras y anótalo en tu hoja de personaje. 

Escoge también el género de tu personaje, masculino o femenino o el que prefieras, no es obligatorio que tu personaje tenga tu mismo sexo, ni que se corresponda con su apariencia.

\section{Atributos}

\subsection{Definición}

Los atributos\index{atributos} definen las capacidades de los personajes. En este juego todos los personajes, jugadores o no jugadores, tienen siete atributos; todos ellos muy útiles para ir de aventuras.

\begin{description}
\item[Fuerza:]
Mide \index{fuerza}lo fuerte que eres. Usamos el atributo de Fuerza\index{fuerza} para comprobar si puedes levantar algo muy pesado, romper cosas o defenderte de algún bruto.

\item[Salud:]
Mide\index{salud} la resistencia de tu personaje a las enfermedades, la fatiga, los venenos y los tartazos con tartas de crema. Lo de los tartazos no es broma.

\item[Agilidad:]
\index{agilidad}Nos indica la facilidad y gracia de los movimientos de tu personaje; sirve para saltar, hacer atletismo, montar una bicicleta, correr o esconderte de una banda de maleantes.

\item[Destreza:]
La destreza\index{destreza} indica lo hábil que es tu personaje con las manos. Eres diestro si tienes una letra bonita, dibujas bien o eres capaz de pilotar un dirigible. También sirve para reparar un motor o, por supuesto, manejar una espada.

\item[Educación:]
Mide todo lo que tu personaje sabe del mundo. El atributo de Educación\index{educación} rige todo lo que puede aprenderse y conocerse; aunque no sea en un colegio. Sirve para conocer datos históricos y geográficos, desde luego, pero también para cosas más prácticas como saber cómo vendar una herida o quién vende las mejores habichuelas de este condado.

\item[Ojos y Oídos:]
Mide la capacidad que tiene tu personaje de entender lo que ocurre a su alrededor. Aunque lo llamemos Ojos y Oídos\index{ojos y oídos}, en realidad incluye los cinco sentidos de tu personaje; lo que pasa es que Ojos y Oídos son los órganos que normalmente se usan más en las partidas, cuando intentas buscar un mensaje secreto escondido entre los cascotes de unas ruinas o discernir si los pasos que escuchan a tu espalda provienen de un gracioso gato o un taimado espía.

\item[Carisma:]
Este atributo mide en un solo valor lo mono, simpático, agradable, guapo, alegre y atractivo que es tu personaje. El atributo de Carisma\index{carisma} te vendrá muy bien para vender periódicos, hablar un idioma extranjero, convencer a un caníbal de que sabes muy mal, mendigar, tocar un instrumento musical, cantar o actuar tan bien que la gente te pague por ello.

\end{description}

Los atributos se miden en dados, que van a ser los dados que lances cuando necesites saber si tu personaje consigue o no lo que se propone. 

\begin{description}
\item[Ejemplo:] Pepe tiene 3 \textsc{dn} (dados normales) en Carisma y quiere comprarse una bicicleta vieja que cuesta \$20, pero sólo tiene \$15. Intenta convencer al dueño de que le haga una rebaja usando su Carisma. El Director de Juego le dice que vale, pero que tiene que sacar un 15 con los dados, que es bastante difícil. Pepe lo intenta y saca un 4, un 3 y un 2, por un total de 9. ¡Mala suerte! Ha sacado menos y el dueño no acepta hacerle ninguna rebaja.
\end{description}

Cada atributo tiene que tener asignado al menos un dado; sólo algunos monstruos carecen de algún atributo --por ejemplo los zombies no tienen Carisma de feos y tontos que son-- y entre más y mejores dados tengas mejor será tu personaje en ese atributo.

\section{Generando tus atributos}

Tus atributos necesitan dados, ¿cómo los consigues? Pagándolos con nuestros maravillosos \textsc{Puntos de Atributos} o \textsc{PA} para los amigos. Si tu personaje es menor de 14 años, tienes tantos \textsc{PA} como tu edad más nueve. Si es mayor, lo más fácil es que consultes el cuadro, ya que la regla sería complicada de explicar. 

\begin{table}[h]
\centering
\begin{tabular}{l c c c c c c}
\toprule
Edad&9 a 14&15 a 17&18 a 25&26 a 40&41 a 60&60 ó más\\\midrule
\textsc{PA}&Edad+9&24&25&24&23&22\\\midrule
\bottomrule
\end{tabular}
\caption{Puntos de Atributo}
\end{table}

Con estos Puntos de Atributo puedes \emph{comprar} dados para los atributos de tu personaje. Cada dado débil cuesta 1 \textsc{PA}, y los dados normales cuestan 2 \textsc{PA}. También puedes convertir un dado débil en un dado normal por 1 \textsc{PA}.

Todos los atributos tienen que tener al menos un dado, ya sea fuerte o débil. En cada atributo puedes comprar hasta cuatro dados, ya sean fuertes o débiles.

Como los dados normales no dan el doble de puntuación que los dados débiles pero cuestan el doble, es mejor que compres cuatro dedos débiles en un atributo y luego, si te quedan puntos, conviertas alguno de esos dados débiles en dados normales.

Por si te lo preguntabas, no puedes comprar dados malos ni dados fuertes cuando creas tu personaje; ya verás como se usan estos dados cuando lleguemos al capítulo que explica el sistema de juego.

\section{Pillerías}

Las \textsc{pillerías} son puntos que puedes usar para cambiar la suerte de tu personaje. Por ejemplo, puedes pedir que te dejen tirar otra vez los dados, evitar la muerte o incluso hacer que aparezcan cosas agradables para ti. Eso sí, una vez gastados las pillerías son difíciles de recuperar, así que deberás gastarlos con cuidado.

Ahora no voy a dar más detalles, porque ya les dedicaré un capítulo entero. Te basta saber cuántas pillerías empiezas y cuántas son las máximas que podrás acumular a lo largo de tus aventuras. Aquí es cuando los niños tienen ventaja, comprueba la tabla, si no me crees.

\begin{table}[h]
\begin{tabular}{lcccccccccc}
\toprule
Edad&9&10&11&12&13&14&15&16&17&Adultos\\\midrule
Pillerías Iniciales&11&10&9&8&7&6&5&4&3&2\\\midrule
Pillerías Máximas&25&24&22&20&17&14&8&6&5&4\\\midrule
\bottomrule
\end{tabular}
\caption{Pillerías Iniciales}
\end{table}

\section{Equipo y Orígenes}

El equipo, las riquezas y el origen de tu personaje depende del \textsc{Escenario de Juego} que deseéis jugar y tendrás que comprobar las reglas que aparecen en el módulo correspondiente.

% Poner referencia
% Si es posible, sacar libretos independientes con el resumen de todo por cada Escenario de Juego