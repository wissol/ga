\capitulo{Introducción}


\minisec{Notas del diseñador}

Hace ya más de cinco años que creé \textsc{Newsies \& Bootblacks}\footnote{<<Vendedores de Periódicos y Limpiabotas>>}, un juego de rol peculiar en el que los personajes tenían entre nueve a catorce años y vivían por su cuenta vendiendo periódicos y superando aventuras en la ciudad \emph{steampunk} de New Paris. Este juego, Grandes Aventuras, es una evolución de ese juego, pero con diferencias muy importantes:

\begin{description}
\item[Un mundo mayor] En Newsies \& Bootblacks, las aventuras se limitaban, casi en su totalidad a la ciudad de New Paris. En Grandes Aventuras, el mundo se hace mucho mayor, y se detallan tres \textsc{escenarios de juego}: las Islas Salvajes de los Mares del Sur y el Salvaje Oeste, además de la propia ciudad de New Paris.
\item[Personajes de todas las edades] Todavía existe la posibilidad de jugar con personajes que sean todos niños y de hecho es la opción estándar para el escenario de New Paris, pero ahora y de serie, los personajes podrán tener cualquier edad.
\item[Un sistema de juego simplificado] Lo habitual, al crear una segunda versión de un juego, es añadir más reglas. Yo he decidido ir por la dirección contraria, simplificando todo lo posible, limpiando el texto y haciendo todo más claro. 
\item[En Español] Newsies \& Bootblacks, como podría suponerse por el título, lo escribí en inglés, lo cual resultó un esfuerzo tremendo, aún con toda la ayuda que recibí. Hoy prefiero dedicar esos esfuerzos a crear un mejor juego.
\end{description}

Quisiera aclarar que este no es un juego de rol para niños, sino un juego que, --desde luego--, puede jugarse con niños, pero que puede admitir jugar con otras personas. A diferencia de su primera encarnación, GA, permite personajes jugadores adultos y niños. No debe llevarse a la idea de que un PJ niño debe jugarse por un niño, ni un adulto por un adulto. De hecho, la opción contraria suele ser mucho más divertida, sobre todo si se juega en el ámbito familiar.

\minisec{Una partidida}

Nadie aprende a jugar al fútbol leyendo el libro de reglas. Si acaso te acuerdas de los tiempos en los que te bamboleabas intentado andar, recordarás que tus amigos jugaban a la pelota pero no se valía darle con la mano y tenías que meter la pelota entre tres palos. Luego jugabas y te enterabas de que no podías pegar patadas y había una cosa muy rara, que todo el mundo se hacía un lío, que se llamaba fuera de juego.

También la mejor forma de aprender a jugar a un juego narrativo, o de rol, es unirte a gente que ya juegue. Sin embargo, es posible que no tengas esa suerte; de hecho eso fue lo que me pasó a mí. Para ese caso he previsto la segunda mejor solución: dejarte ver cómo sería una sesión de juego. Venga, ponte cómodo, aquí todos somos familia.

Martín, Ana, Jaime y Daniel están sentados a una mesa, jugando a una partida de \textsc{¡Grandes Aventuras!} en New Paris, por supuesto. Martín, el Director de Juego, se sienta en un extremo, con una pila de folios donde ha escrito una aventura, el libro de reglas y un puñado de dados. Los demás tienen un par de folios, en uno pone \textsc{Hoja de Personaje}, donde está escrito lo que cada personaje puede hacer y las cosas que tiene, la otra es para notas. También tienen dados, sólo para ir más rápidos. En el centro hay un mapa de las alcantarillas de New Paris que Martín ha dibujado y donde se desarrolla la aventura.

\begin{quotation}

Martín: \enquote{Vale, ¿váis a perseguir a ese landrozuelo por las alcantarillas o no? Tenéis cinco segundos.}

Ana: \enquote{Yo digo que vamos, quiero recuperar mis cosas}. El personaje de Ana es una joven vendedora de doce años que no sabe que es el miedo, ni la prudencia.

Daniel: \enquote{Pues, no sé... podían haber cocodrilos, creo}. ---El personaje de Daniel es un flacucho de nueve años.

Jaime: \enquote{Cocodrilos, sí hombre, no me digas, vamos pa'dentro}. ---Jaime lleva a un brutote de trece años.

Daniel: \enquote{¿Y el fantasma que vimos ayer?}

Ana: \enquote{No sabes si fue un fantasma, ¿verdad? Y aunque lo hubiera sido, da igual, le ganamos}.

Martín: \enquote{Menos cháchara. ¿Váis o no? El chico se está escapando mientras perdéis el tiempo}.

Ana y Jaime: \enquote{¡Vamos!}

Daniel: \enquote{Vale pero si nos matan es culpa vuestra.}

Martín: \enquote{Muy bien. Entráis en la penumbra tenebrosa de la alcantarilla. El aire, pútrido y enrarecido, se cuela en vuestros pulmones mientras vuestros ojos luchan por ver dónde pisáis. Por cierto... ¿alguien tiene una linterna?}

Daniel: \enquote{No.}

Ana: \enquote{Yo traje una vela; la enciendo}

Martín: \enquote{Vale, quiero que todos hagáis una prueba de \enquote{Ojos y Oídos}. Tirad los dados, el \enquote{Número Objetivo} es 13}.

Daniel:\enquote{Saco 14, ¿estoy bien?}

Martín: \enquote{Sí.}

Ana: \enquote{Vaya, saqué un 12, ¡por uno!.}

Jaime: \enquote{Y yo un 8, son estos dados, que son viejos}

Martín: \enquote{Vale. Daniel, tú te paras justo a tiempo, pero Ana y Jaime se caen en una corriente de agua marroncilla que les llega hasta los codos. Ya sabéis que clase de agua es esta, ¿verdad?} --- Daniel se parte de risa.

Martín: \enquote{Daniel, por cierto, podrías haber escuchado algo a tu espalda. Haz otra prueba de Observación, por favor. El número objetivo es 11 esta vez}.

Daniel: \enquote{Vale... sin problemas, tengo tres dados malos pero soy bueno en observación, así que los mejoro. Tres dados normales, gracias.} Daniel tira los dados: \enquote{4,6,3, suman trece, ¿qué tal?}

Martín: \enquote{Mayor que el Número Objetivo, genial. Genial, Daniel, tan pronto como te vuelves descubres a una rata escabuyéndose entre las sombras}.

Daniel: \enquote{¿Era eso, sólo una rata?}

Martín: \enquote{Bueno, sí, si quieres la cambio por un zombie}.

Daniel: \enquote{¡No-o!}

Ana: \enquote{Bueno, Martín ¿y qué hacemos ahora?}

Martín: \enquote{Lo que queráis, como siempre. Sugiero que salgáis de esa… eh… bueno… corriente y decidid si queréis ir rumbo norte o sur.}

Jaime: \enquote{Vale, lo hacemos. ¿Dejó el ladrón alguna pista?}

Martín: \enquote{Sí, desgraciadamente hay demasiada oscuridad en la alcantarilla}.

Daniel: \enquote{Da igual, lo intento}.

Martín: \enquote{Vale, el número objetivo es 16; más te vale tener suerte.}

Daniel lanza los dados: \enquote{¡Todo seis! ¡Soy el mejor!}

Jaime: \enquote{Suerte es lo que tienes}.

Martín: \enquote{Un éxito espectacular, de pura chiripa descubre los sutiles brillos de 25 monedas de un centavo extendidas en línea hacia el sur.}

Ana: \enquote{Me encanta. Pillamos el dinero y vamos al sur.}

Jaime: \enquote{Espera, mejor nos hacemos antorchas primero. Martín, ¿hay algo por aquí con lo que podamos hacer antorchas?}

Martín: \enquote{Eso sería mucha suerte}.

Jaime: \enquote{Que lo intente Daniel}.

\end{quotation}

\minisec{El director de juego y los jugadores}

Como ves la primera regla para jugar a un juego narrativo o de rol es reunir a un grupo de amigos en torno a la mesa. El número mínimo es 2: porque debe existir al menos 1 director de juego que cuenta la historia, lleva a los personajes secundarios y hace de árbitro. El resto de los jugadores lleva a sus propios personajes, los personajes jugadores o PJ, decidiendo sus acciones libremente.

El juego narrativo, o de rol, es una mezcla de narración, estrategia, cooperación y suerte. La narración, lo que va contando el Director de Juego, te permite conocer la situación del mundo. Normalmente, el Director del Juego querrá apoyarse en un mapa, ilustraciones o incluso usar música de fondo para introducirte más en el juego; pero todo eso son herramientas opcionales. Ante la narración del Director de Juego, los jugadores pueden intentar hacer lo que quieran. Es aquí donde la estrategia y la cooperación son imprescindibles dado que casi siempre los jugadores tendrán que elaborar y ejecutar un plan en conjunto. Lo menos importante, diga lo que diga Jaime, es la suerte. Sí, ayuda sacar buenas tiradas, pero el Director de Juego no asigna los números al tuntún, sino que usa Números Objetivo altos para cosas difíciles y bajos para lo fácil. Una buena estrategia te permitirá que cada personaje ayude haciendo lo que sea más fácil para él.

Es raro que los héroes sean derrotados en una película. Puede que les den una tunda en la primera escena o que cosechen fracaso tras fracaso, pero al final, lo normal es que triunfen. En los juegos de rol los finales no serán siempre felices y eso es lo que da emoción al juego. Si eres listo, juegas bien y tienes suerte ganarás, pero cualquier error puede tener terribles consecuencias para tu personaje.
Por eso tenemos Directores de Juego, que es una combinación de árbitro y narrador. El Director de Juego ha de conocer bien la historia y las reglas para poder aplicarlas con justicia a cada caso. Su decisión es definitiva. El Director de Juego también está encargado de llevar el mundo de juego: el clima, los sucesos que ocurran y los personajes no jugadores.

Sé por propia experiencia que ser Director de Juego requiere dedicación y empeño, como todas las cosas que valen la pena. Pero si lo haces bien tendrás la misma satisfacción que la de un director de cine, aunque tu audiencia sean sólo tus amigos.

Quizás, precisamente, lo mejor es que a quienes se los está haciendo pasar bien es a tu amigos.

\minisec{Un mundo de juego}%-- cambiar

Todo juego narrativo necesita un mundo de juego; en \textsc{Grandes Aventuras} vivimos una época que solo existió en los sueños del Capitán Robert, mezcla de revolución industrial, gran desarrollo económico, imperios coloniales y una onza de fantasía. Además, he desarrollado especialmente tres \textsc{Escenarios de Juego}. 

\begin{description}

\item{New Paris} 
Una metrópolis situada en algún lugar de la costa este de Estados Unidos y bulle con malvados espías, políticos corruptos, científicos locos, revolucionarios, soñadores, alcantarillas con oscuros secretos, extrañas criaturas y puertas secretas, y, por supuesto, los \emph{Newsies}, jóvenes vendedores de periódicos que sobreviven por su cuenta y descubren grandes aventuras.

\item{Las Islas Salvajes de los Mares del Sur}
El Oceáno Pacífico, al norte de Indonesia, oculta en su interior los Mares Perdidos y sus miles de islas vírgines, llenas de tesoros, antiguas ruinas, piratas y caníbales.

\item{El Salvaje Oeste}
El sueño de muchos \emph{Newsies} es llegar aquí algún día, ya sea en los \emph{trenes de huérfanos}, ya sea cuando se hagan mayores, para probar fortuna y fundar una familia o simplemente vagar de aventura en aventura a lomos de un buen caballo.

\end{description}



Pero lo esencial no es el mundo de juego, ni la fantasía, ni las estrategias, ni los dados, la historia, ni siquiera los directores de juego, sino los héroes.

\minisec{Héroes}

Hace mucho, mucho tiempo, en un internado... Conocí a un chiquito de primaria, de esos menudos trabajadores que se esfuerzan por ser mejores. Un día, en el dormitorio, uno de sus amigos le preguntó: \enquote{J, ¿por qué estudias tanto?} Su respuesta fue, \enquote{porque me siento muy bien cuando me voy a la cama.} Los otros niños suspiraron de alivio y uno de ellos dijo: \enquote{Eres un vago, como nosotros}. Nuestro héroe sólo respondió con una cara que decía: \enquote{no lo pillan}. Yo sí, es la satisfacción del deber cumplido, que no se parece ni puede cambiarse por ningún otro tipo de felicidad, diga la publicidad lo que quiera. Quien hace de la vida un camino para ser mejor, sin obsesiones ni miedos, es un héroe. Y éste es el tipo de héroe de ¡Pillastres!

\minisec{Aventuras}

Los héroes necesitan aventuras. Supón que tienes 12 años, son las cuatro de la mañana y te diriges a la Avenida de los Periódicos; todavía quedan ladronzuelos, matones y gente más siniestra escondida entre la niebla y la nevisca. Toda tu esperanza está en tus bolsillos: unos cuantos pequines con los que comprarás un lote de periódicos, que luego revenderás, uno a uno, a cambio de un pequeño margen de beneficio.

Ésta era la vida normal de muchos niños de finales del XIX y principios del XX, y peores días habían tenido. Pero ahora le podemos añadir un pequeño fantasma simpático y, quizás, su madre fantasma quien, mira por dónde, no es tan simpática. Tal vez, sólo tal vez, un malvado espía extranjero te persiga; ¿estás seguro de conocer a tus verdaderos padres? Y... ese tipo con bata blanca que te mira fijamente, ¿será un cliente llamando tu atención o un científico loco buscando un aprendiz o una cobaya humana? Y quizás todo eso sean fantasías y cuentos de viejas, pero los camorristas del distrito Decree son bien reales y te están sonriendo, aunque no como lo haría tu madre, precisamente...

\minisec{¿Quién gana?}

Bueno, esta es difícil, porque los juegos narrativos pueden ganarse y perderse de muchas maneras. En primer lugar no hay una puntuación y no juegas contra nadie. En ese sentido no hay perdedores ni ganadores. Si habéis vivido una gran aventura y os lo habéis pasado bien, habéis ganado.

Pero... sin embargo, todas las aventuras de ¡Pillastres! tendrán un objetivo: algo que los personajes jugadores querrán conseguir o evitar. Algunas veces el objetivo será evidente, como rescatar a un amigo o descubrir un tesoro, pero otras el Director de Juego mantendrá el objetivo principal oculto durante la mayor parte de la aventura, o incluso cambiarlo a la mitad.

¿Qué tiene de bueno cumplir los objetivos? Bueno, además de sentirte ganador, es probable que tu personaje reciba alguna recompensa. Pero recuerda, es un esfuerzo de equipo; no puedes ganar si tus amigos pierde; ni en un juego de rol ni tampoco en la vida real.

\minisec{Preparado para las reglas}

Espero que ahora tengas una pequeña impresión de lo que es esta clase de juegos. No te preocupes, ya lo irás pillando. Yo empecé con un libro de reglas en los tiempos en que no había Internet y ninguno de mis amigos había oído hablar jamás de los juegos de rol. Te las apañarás muy bien, y recuerda, no pasa nada si te olvidas de alguna regla al principio; le pasa a todo el mundo.