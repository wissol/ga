\capitulo{Peleas y Persecuciones}

Grandes Aventuras no es un juego de combates. Sin embargo en las
aventuras a veces se encuentran con gentes dispuestas a practicar
la violencia si es conveniente a sus fines. Además, las consecuencias
pueden ser definitivas para los personajes y la aventura, por lo que
tenemos que dedicarle cierta atención.

\section{Orden en las acciones}

En el combate primero mueven por el siguiente orden.

\begin{enumerate}
\item Personajes Jugadores
\item Personajes no Jugadores Hostiles
\item Personajes no Jugadores Neutrales
\item Personajes no Jugadores Aliados
\end{enumerate}

Los personajes no jugadores se consideran hostiles, neutrales o aliados conforme 
a cómo se hayan comportado hasta ahora. Si un pnj aliado decide traicionar a los pj's
deberán mover en el orden de los pnj aliado hasta el mismo turno en que se revele 
hostil; --apuñalando por la espalda, por ejemplo\footnote{Esta circunstancia debería ser
rara, de lo contrario los jugadores acabarán desconfiando del máster}.

\section{Secuencia}

\begin{enumerate}
\item Sorpresa (Si es necesario)
\item Disparo
\item Movimiento
\item Compate
\item Victoria
\end{enumerate}

\section{Sorpresa}

Esta fase de la secuencia de combate solo se usa en el primer turno de un
combate y solo si uno de los dos bandos que combaten no habían detectado
a al menos alguno de los miembros del otro bando; como en una emboscada.

Para determinar si hay emboscada el bando que puede caer en la emboscada 
designa a uno de sus miembros para que haga una tirada de \emph{Ojos y Oídos}.
El otro bando hace ahora una tirada de \emph{Agilidad}. Los que saquen tanto
o más que la tirada de Ojos y Oídos habrán sorprendido al enemigo.

Los que hayan sorprendido al enemigo serán los únicos que podrán realizar acciones
durante el resto de este turno. Una vez se termine el turno todos los personajes
podrán actuar normalmente sin que se vuelvan a usar las reglas de sorpresa. (Excepto
en el extraño caso que alguien se incorpore sigilosamente al combate cierto tiempo
después.)

\section{Disparos}

Los personajes mueven por el siguiente orden de prioridad.



% \section{Shooting, throwing and ranged attacks }
% \begin{enumerate}
% \item Attacks by shooting are resolved just like any other skill check,
% using the appropriate skill. 
% \item In \emph{Newsies \& Bootblacks} all weapons are considered to be in
% range%
% \footnote{You can always try to hit, no matter how far away the target is.%
% }. 
% \end{enumerate}

% \subsection{Target number}
% \begin{enumerate}
% \item For most weapons, the TN is 6 + 1 for every full 10 yards of distance. 
% \item For short-range weapons, the TN is 6 + 1 for every five full yards
% of distance. 
% \item For extreme short-range weapons the TN is 4 + 1 for every two full
% yards of distance. 
% \item Add 1 if the target is crouching or taking cover under a table, behind
% a low wall or something like that. 
% \item Add 2 if the target is smaller than a head. (Or, if you can only see
% the head of your target.)
% \item Add 4 if you cannot see your target but it's still possible to hit
% the target; lobbing a stone, for example.
% \item The TN cannot get any higher than 18, nor any lower than 6.\end{enumerate}
% \begin{quote}
% Example: Pip wants to hit a can that is 30 yards away, using his slingshot.
% The slingshot is a short-range weapon, so the TN for this roll is
% 6 + (30/5) + 2 = 6 + 6 + 2 = 14. Had he tried to do the same with
% a huge rock, an extreme short-range weapon, the TN would have been
% 4 + (30/2) + 2 = 4 + 15 + 2 = 21; which is cut down to 18 (the maximum
% TN as per point 7 above). Wish him luck.
% \end{quote}

% \subsection{Dodging ranged attacks}
% \begin{enumerate}
% \item All missile and thrown attacks can be dodged%
% \footnote{Yes, this includes bullets and ninja stars.%
% }, \emph{provided the character is aware of the attack and can see
% his attacker. }
% \item The target number to dodge an attack is the number the attacker rolled.
% \item Double the target number if dodging a bullet.
% \item Each attack can be attempted to dodge just once.
% \item There is no limit to the number of different attacks a \emph{player
% character} can attempt to dodge.
% \item A non-player character can only attempt to dodge as many attacks as
% the number of dice of his Agility attribute. (So a policeman with
% an AGI score of 4 wd, can attempt to dodge up to 4 attacks per turn.)
% \end{enumerate}

% \section{Movement and reaction phase}

% Every character, animal or vehicle in this game moves one pace per
% turn at their normal speed. The trick lies in that some paces are
% longer than others. For anybody younger than nine; one pace equals
% two yards per turn; for nine to 11, that's three yards per turn; for
% 12 to 14 it's four yards per turn, while those 15 years of age and
% older move at five yards per turn. The following table demonstrates
% some examples. The game master may adapt those for anything not covered
% here, such as giant apes or aliens from outer space. 

% That explains your basic walking movement, but there could be times
% when you want to go much faster. In those cases you have to roll on
% Athletics, Riding or Vehicles. Your choice depends on whether you
% are on your feet, riding a horse or using some modern vehicle. The
% target number should be determined by the game master, taking into
% account the state of the ground, the weather and any other appropriate
% circumstances. If everything is normal the TN should be 10. 

% After passing the test, the character doubles his pace%
% \footnote{Or that of the vehicle he's using.%
% }. Triple the pace if the character achieves a spectacular success.

% However, if you fumble you only move half your pace and then suffer
% a mishap. If you are running or riding a bicycle when you suffer the
% mishap, you trip or fall to the ground, possibly getting a few bruises.
% Inside a vehicle, both driver and passengers could suffer life-threatening
% wounds%
% \footnote{Details can be found in Section \vref{sec:Damage}.%
% }. 

% \noindent \begin{center}
% \textbf{\large Movement Table}
% \par\end{center}{\large \par}

% \begin{center}
% \begin{tabular}{cc}
% \toprule 
% You are / You are using & Pace\tabularnewline
% \midrule
% \midrule 
% 9 to 11 years old & 3 yards\tabularnewline
% \midrule 
% 12 to 14 years old & 4 yards\tabularnewline
% \midrule 
% 15 and older  & 5 yards\tabularnewline
% \midrule 
% Swimming & Take 2 from your usual pace\tabularnewline
% \midrule 
% Bicycle & Double your usual pace\tabularnewline
% \midrule 
% Horseless carriage & 13 yards\tabularnewline
% \midrule 
% A steam tank & 4 yards\tabularnewline
% \midrule 
% Light horse carriage & 14 yards\tabularnewline
% \midrule 
% Horse & 15 yards\tabularnewline
% \midrule 
% Heavy horse carriage & 10 yards\tabularnewline
% \bottomrule
% \end{tabular}
% \par\end{center}


% \section{Close combat}
% \begin{enumerate}
% \item Every character can only attack once per turn%
% \footnote{Except in very special cases, as determined by the game master.%
% }.
% \item A character who is facing an enemy at less than \emph{two yards} or
% so, can attack in close combat.
% \item The attacker rolls for the appropriate \emph{weapon,} skill, or Fisticuffs,
% if fighting unarmed.
% \item The attacker should add or subtract the appropriate modifiers that
% the game master imposes on the roll, to provide for special circumstances.
% A table with examples follows.
% \item The defender rolls, using his Dodging skill.
% \item If the defender rolls higher or equal to the attacker, the attack
% has failed.
% \item If the attacker rolls higher than the defender, the attack has been
% successful and the defender will probably suffer some form of harm.
% \end{enumerate}
% \noindent \begin{center}
% \textbf{\large Close Combat Table}
% \par\end{center}{\large \par}

% \begin{center}
% \begin{tabular}{cc}
% \toprule 
% Circumstance & Modifier to Attack Roll\tabularnewline
% \midrule
% \midrule 
% Low light & - 1\tabularnewline
% \midrule 
% Restrained space & - 2\tabularnewline
% \midrule 
% Defender in higher ground & - 2\tabularnewline
% \midrule 
% Defender surprised (first round) & + 3\tabularnewline
% \midrule 
% Attacking from the rear & + 5\tabularnewline
% \midrule 
% Attacking from the side & + 1\tabularnewline
% \bottomrule
% \end{tabular}
% \par\end{center}

\section{Daño \label{sec:Daño}}

Si golpean a tu personaje lo más probable es que sufra alguna clase
de daño o incluso la muerte. El combate en este juego es \emph{muy peligroso},
--mucho más que en otros juegos de rol--, así que no debería suceder a menudo
y si pasa y no puedes evitarlo, ten cuidado, mucho cuidado.

\subsection{La tirada de daño}

La idea básica es que haces una tirada de salud contra un \# que depende del
arma empleada, si aciertas sufrirás menos daño que si fallas; los detalles 
dependen tambien de la clase de daño del arma.

\minisec{Armas de Tipo E o "Juguetes"}

No son verdaderas armas sino elementos que podrían usarse en una bronca más o menos
amistosa, sobre todo entre niños. Nunca usarías un arma de esta clase en un combate
serio. Estamos hablando de tirachinas y tartas de crema.

\begin{description}

\item[Éxito:] Si pasas la tirada tu personaje no sufre ninguna consecuencia negativa.

\item[Fallo:] Si fallas, tu personaje sufre un \emph{punto de golpe verde, pg verde, o simplemente verde}. 
Cada \emph{pg verde} supone una penalización de -1 a las tiradas de todas tus acciones. Esta penalización
se acumular al recibir más golpes, pero desaparece tan pronto termine la riña o degenere en un combate verdadero,
como cuando empezáis una \emph{batalla} de tartas de crema y alguien saca un revólver. 

\end{description}

\minisec{Armas de Tipo D o "No letales"}

En esta clase de armas son las que, normalmente, no matarían a un adulto en una novela
o una película pero que pueden romper algún hueso o herir. Estamos hablando de bastones, 
piedras, botellas y cosas de ese estilo. 

\begin{description}

\item[Éxito:] Si pasas la tirada tu personaje no sufre ninguna consecuencia negativa.

\item[Fallo:] Si fallas, tu personaje sufre un \emph{punto de golpe rojo, pg rojo, o simplemente rojo}. 
Cada \emph{pg rojo} supone una penalización de -1 a las tiradas de todas tus acciones. Esta penalización
se acumular al recibir más golpes. A diferencia de los verdes, los rojos no desaparecen cuando termina 
el combate.

\end{description}

\minisec{Armas de tipo C o "letales"}

En esta clase de armas son las que podrían llegar a matar a un adulto en una novela
o una película pero que pueden romper algún hueso o herir. Estamos hablando de cuchillos, los dientes
de un perro y armas de calibre muy pequeño.

\begin{description}

\item[Éxito:] Tu personaje aguanta bien la herida y sólo sufre 1 \emph{pg}

\item[Fallo:] Tu personaje recibe una herida importante, moviéndose 1 nivel en la escala de daño.

\end{description}

\minisec{Armas de tipo B o "militares"}

Estas armas normalmente matarían a un adulto, tanto en la ficción como en la vida real. Aquí se incluyen
la mayoría de las armas de fuego, las espadas y las lanzas.

\begin{description}

\item[Éxito:] Tu personaje recibe una herida importante, moviéndose 1 nivel en la escala de daño.

\item[Fallo:] Tu personaje muere.

\end{description}

\minisec{Armas de tipo A o "artillería"}

Las armas de tipo A usan las mismas reglas que las de tipo B, con la salvedad de que pueden también dañar
vehículos y estructuras como edificios y puentes. Aquí estamos hablando de cañones y explosivos.


\begin{table}[ht]
\centering
\begin{tabular}{lccccl}
\toprule
Tipo de daño  & Éxito & Fracaso \\\midrule\midrule
E             &  -    & -1 verde \\\midrule
D             &  -    & -1 rojo\\\midrule
C             & -1 rojo  & Herido  \\\midrule
B             & Herido & Muerte \\\midrule
A             & Herido & Muerte \\\midrule
\bottomrule
\end{tabular}
\caption{Resumen de tirada de daño}
\end{table}
% \subsection{The damage roll}
% \begin{itemize}
% \item Test your Health against the damage rank of the weapon that hit you.
% \item The damage rank of unarmed attacks is equal to a roll of the attacker's
% Strength dice (STR)%
% \footnote{Pip's Strength is three weak dice, so when he hits an opponent he'll
% roll three weak dice; the total being the damage rank of the attack.%
% }.
% \item The damage rank of most hand-to-hand weapons is equal to a roll of
% the attacker's Strength dice plus some modifier.
% \item The damage rank of most missile weapons is a fixed number.
% \item The effects depend on the nature of the weapon and your roll.\end{itemize}
% \begin{description}




% \begin{quote}
% \emph{For example:} you are involved in a fight with a band of bullies.
% They hit you seven times in the face (ouch!). That means you have
% to test your Health seven times. You are lucky (or tough) enough to
% pass five of these checks. But you failed two of these so you received
% two bruises. Now you are at \emph{-}2, so every time you roll the
% dice to check for a skill or attribute, you subtract 2 points to every
% roll. If you later receive another bruise, you'll be at \emph{-}3
% and you'll have to subtract 3 points to every roll, and so on. 
% \end{quote}
% \item [{Fumble:}] If you fumble the check you move down one step in the
% damage scale. This is explained in Section 6.7.2. It means you go
% from swell to hurt, from hurt to grave and from grave to dead. 
% \end{description}
% \item [{Toy~Weapons:}] Toys that cannot kill an adult%
% \footnote{In real life silly accidents are always possible.%
% }. In fact, most may only cause embarrassment. We are speaking of such
% terrible instruments as cakes, splashing of water and such. 


% 

% \begin{description}
% \item [{Spectacular~Success:}] Luckily, the attack was too weak and your
% character is unharmed.
% \item [{Success:}] If your character passes the Health check he is still
% bruised and you receive a -\,1 bruise. 
% \item [{Fail:}] If your character fails the Health check he moves down
% one step in the damage scale that moves in succession from swell to
% hurt to grave and finally, to dead%
% \footnote{More details are found in section \ref{sub:TDamage-Scale.}. %
% }. 
% \item [{Fumble:}] If your character fumbles he will die unless you spend
% a chit right away. 
% \end{description}
% \item [{Extremely~Lethal~Weapons:}] Weapons that can kill an adult in
% one single attack, and often do so in stories. Examples include swords,
% revolvers, rifles or large fires. You are advised to be very careful
% when somebody uses them. 

% \begin{description}
% \item [{Spectacular~Success:}] Even with a spectacular success your character
% receives one bruise. Consider it a scratch.
% \item [{Success:}] If your character passes the Health check he moves up
% one step in the damage scale.
% \item [{Fail~or~Fumble:}] If your character fails or fumbles he will
% die unless you spend a chit right away. 
% \end{description}
% \end{description}

\begin{table}[ht]
\begin{tabular}{lccccl}
\toprule
Arma               & Daño       & \# Salud  & Alcance   & Munición  & Notas             \\\midrule\midrule
Humano desarmado   & D          & FUE       & -         &           &                   \\\midrule
Palo o Bastón      & D          & FUE +1    & E         &           &                   \\\midrule
Piedra o Ladrillo  & D          & FUE +2    & D         &           &                   \\\midrule
Gran roca          & C          & FUE +3    & -         &           & \#+3              \\\midrule
Cuchillo           & C          & FUE +1    & E         &           &                   \\\midrule
Espada             & B          & FUE +3    & E         &           &                   \\\midrule
Ilo utala          & C          & FUE +4    & E         &           & Arma nativa, 2 manos       \\\midrule
Lanza              & B          & FUE +2    & D         &           &                    \\\midrule
Derringer          & C          & 13        & D         & 1         & \#+1              \\\midrule
Revólver           & C          & 15        & C         & 6         &                   \\\midrule
Escopeta           & B          & 18        & C         & 2         & \#-1              \\\midrule
Rifle              & B          & 16        & B         & 1 ó 6     &                   \\\midrule
Carabina           & C          & 15        & B         & 1 ó 6     &                    \\\midrule
Cerbatana          & E          & 9         & D         & 1         & Puede llevar veneno \\\midrule
Tirachinas         & E          & 10        & D         & 1         &                   \\\midrule
Tarta de Crema     & E          & 16        & E         &           &                   \\\midrule
Tomates            & E          & 12        & E         &           & \\\midrule
Periódico enrrollado & E        & FUE       & -         &           & \\\midrule
\bottomrule
\end{tabular}
\caption{Armas}
\end{table}

\begin{table}[ht]
\begin{tabular}{lccl}
\toprule
Descripción     & Daño & \# Salud & Notas         \\\midrule\midrule
Perro mediano   & D    & 10       &               \\\midrule
Lobo            & C    & 10       &               \\\midrule
Wawa akesi      & C    & 12       & Felino nativo \\\midrule 
pipi suli       & D    & 9        & Mosquito gigante \\\midrule
\end{tabular} 
\caption{Daño de algunos animales}
\end{table}

% Alcance 
% E: Corto, -1 al daño si se lanza, 
% D corto, 
% C normal, 
% B largo, 
% A extremadamente largo

% Daño 
% E: Juguete, 
% D No letal, 
% C letal, 
% B extremo, 
% A destructivo

\begin{table}[ht]
\begin{tabular}{lccl}
\toprule
Descripción     & Daño & \# Salud       & Notas         \\\midrule\midrule
Caída           & C    & 2 por metro    &               \\\midrule
Fuego pequeño   & C    & 13             &               \\\midrule
Fuego grande    & B    & 13             & Por minuto de exposición \\\midrule
Atropello       & B    & 13             &               \\\midrule
Accidente de bicicleta & D  & 12        &               \\\midrule
\end{tabular} 
\caption{Otras fuentes de daño}
\end{table}




% \subsubsection{Special weapons}

% Some weapons are so special they cannot be easily included in a table.
% Let me introduce you to the following:
% \begin{description}
% \item [{Cane-Sword:}] A sword in a cane. With just a turn of your wrist
% on the head of the cane, you will have a sword on your hands.
% \item [{Pepperbox~Pistol:}] The grandfather of the revolver. The pepperbox
% pistol has several individually loaded barrels that rotate as the
% trigger is pulled. Treat it as a revolver.
% \item [{``My~Friend'':}] A knuckle-duster pepperbox combo. This odd
% weapon is shaped so it can easily be held and used as a knuckle-duster.
% \item [{Gaulois~Pistol:}] The Gaulois pistol looks like an elaborate cigar
% case of A-grade, yet it's intended for self-defense at close range.
% Use it as a revolver, but lower the range to extreme short.
% \item [{Penknife~Pistol:}] A penknife and pistol combination. Treat it
% as a derringer when shooting. If used in hand-to-hand combat, treat
% it as a cane.
% \item [{Walking-Stick~Gun:}] A carbine disguised as a stick. Use it as
% a rifle, but lower the damage to 13.
% \end{description}

% \subsection{The damage scale\label{sub:TDamage-Scale.}}

% The damage scale is an easy way to say how hurt a character is. Every
% player starts at swell and during game time, they may be downgraded
% to hurt. From hurt a player worsens to grave, which is a serious condition,
% and then finally, dead. Each of these conditions have an effect on
% what a character can do. An explanation of these follows. 
% \begin{description}
% \item [{Swell:}] Your character is perfectly healthy and you have nothing
% to worry about. Bruises are kept from the damage scale, so no matter
% how many bruises you get, you retain your swell status. You can be
% swell and still be at -\,­5 bruises or worse. 
% \item [{Hurt:}] Your character is suffering from a wound or sickness that
% is serious enough to hinder your actions, but not to endanger your
% life. You are at -\,­5 for all your actions (in addition to any bruises).
% So, if you have -\,3 bruises, you are down to -\,­8. 
% \item [{Grave:}] Your character is in real trouble. You are wounded and
% at -\,­5 for all your actions plus any bruises, just as when you
% are hurt. However, it hurts so much you need to roll for Health with
% a TN of 10, just for standing up. Walking requires a TN of 12. Running
% and other strenuous physical activities are next to impossible. 
% \item [{Dead:}] If you don't spend a chit immediately your character is
% quite dead. There is nothing your friends can do about it, except
% to arrange for a decent burial. The game master might let you say
% up to 25 last words. 
% \end{description}

% \subsection{Effects of damage on non-living objects}

% For non-living objects, the game master should assign a fake Health
% attribute. Bruised items will only need a paint job, suffering no
% harm. Hurt items lose one level of quality. Grave items lose two levels
% of quality. Dead items are considered destroyed for all functional
% purposes.

% If an item drops lower than E-grade, it is also destroyed.


% \section{Checking for victory}

% Once all the characters have had the opportunity to act, the turn
% is over. Then the game master should check for victory if any of the
% following conditions have been met.
% \begin{itemize}
% \item The non-player group (and only the non-player group) has been surprised
% in this turn.
% \item It's not a serious fight%
% \footnote{Like in a cake throwing battle.%
% } and the non-player group has suffered more bruises than the player
% character group in this turn.
% \item The non-player group has suffered more hurt (or worse) results than
% the player character group in this turn.
% \item The game master considers that the non-player group has little chance
% to win.
% \end{itemize}
% To check for victory, the game master rolls three normal dice. The
% usual TN is 11, but the game master can increase or decrease this
% number according to the circumstances of the fight.
% \begin{description}
% \item [{Success:}] The non-player group keeps fighting another turn.
% \item [{Fail:}] The non-player group will try to break off combat right
% away, saving their belongings and helping the injured members of their
% group, if any. If the players decide to run after them, uninjured
% non-player characters will defend themselves if attacked; hurt or
% worse non-player characters will surrender as soon as they are in
% close combat range.
% \item [{Fumble:}] The non-player group will run away immediately, as fast
% as they are able, dropping whatever they are carrying and forsaking
% any injured members of their group. If the players decide to run after
% them, the non-player characters will surrender as soon as they are
% in close combat range.
% \end{description}

% \section{\label{sec:Healing}Healing}
% \begin{description}
% \item [{Bruises:}] Healed after a good full night of rest.
% \item [{Bruises~from~Toy~Weapons:}] These are healed immediately after
% the fight is ended. (Write a small ``t'' on your character sheet
% to tell these from regular bruises.)
% \item [{Hurt~or~Grave:}] A hurt or grave character needs some form of
% medical attention. The healer (doctor, nurse or amateur) should roll
% for Healing with a TN of 11 if the character is hurt or 14 if he is
% grave. This roll encompasses all the Healing attempts in any given
% day and can only be tried once a day.

% \begin{description}
% \item [{Fumble:}] The character's condition worsens. Hurt worsens to grave
% and grave worsens to dead (unless the character spends 1 chit).
% \item [{Fail:}] Roll one die. On a roll of 1 the character's condition
% worsens, as if the healer had fumbled. On a roll of 2 to 4 the character
% remains hurt or grave. On a roll of 5 to 6, the character's condition
% improves. Grave worsens to hurt and hurt recovers to swell, but with
% five bruises that will heal normally after a good full night of rest.
% \item [{Success:}] The character's condition improves. Grave improves to
% hurt and hurt recovers to swell, but with five bruises that will heal
% normally after a good full night of rest.
% \item [{Exceptional~Success:}] The character's condition improves. Grave
% recovers to swell, but with five bruises that will heal normally after
% a good full night of rest; hurt recovers immediately.
% \end{description}
% \end{description}
% If the character receives no medical attention, roll one die. On a
% roll of 1 or 2 the character's condition worsens. On a roll of 3 to
% 5 the character remains hurt or grave. On a roll of 6, the character's
% condition improves. Grave improves to hurt and hurt recovers to swell,
% but with five bruises that will heal normally after a good full night
% of rest.